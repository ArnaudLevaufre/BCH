\documentclass[a4paper, 11pt, svgnames]{report}
\usepackage[margin=3cm]{geometry}
\usepackage[T1]{fontenc}
\usepackage[utf8]{inputenc}
\usepackage[french]{babel}
\usepackage{hyperref}
\usepackage[toc,acronym,nopostdot]{glossaries}
\usepackage{graphicx}
\usepackage{float}
\usepackage{minted}
\usepackage{mathtools}
\usepackage[toc, page]{appendix}
\usepackage{subcaption}
\usepackage{rotating}
\usepackage{pst-circ}

\usepackage{tikz}
\usetikzlibrary{decorations.text, arrows, arrows.spaced ,automata, snakes, circuits.logic.US, calc, shapes.geometric}

\tikzstyle{block} = [draw]
\tikzstyle{asyncblock} = [draw]


\pagestyle{empty}
\makeatletter

\tikzset{add font/.code={\expandafter\def\expandafter\tikz@textfont\expandafter{\tikz@textfont#1}}}
\tikzset{port labels/.style={font=\sffamily\scriptsize}}

%
% Default style for components
%

\tikzset{every srr node/.style={draw,minimum width=2cm,minimum height=2cm,inner sep=1mm,outer sep=0pt,cap=round,add font=\sffamily}}
\tikzset{every dff node/.style={draw,minimum width=1cm,minimum height=1.5cm,inner sep=1mm,outer sep=0pt,cap=round,add font=\sffamily}}
\tikzset{every syndrome node/.style={draw,minimum width=5cm,minimum height=2cm,inner sep=1mm,outer sep=0pt,cap=round,add font=\sffamily}}
\tikzset{every utsyndrome node/.style={draw,minimum width=5cm,minimum height=2cm,inner sep=1mm,outer sep=0pt,cap=round,add font=\sffamily}}
\tikzset{every ucsyndrome node/.style={draw,minimum width=5cm,minimum height=2cm,inner sep=1mm,outer sep=0pt,cap=round,add font=\sffamily}}


%
% Shift right register
%

\pgfdeclareshape{srr}{
    % The 'minimum width' and 'minimum height' keys, not the content, determine
    % the size
    \savedanchor\northeast{%
        \pgfmathsetlength\pgf@x{\pgfshapeminwidth}%
        \pgfmathsetlength\pgf@y{\pgfshapeminheight}%
        \pgf@x=0.5\pgf@x
        \pgf@y=0.5\pgf@y
    }
    % This is redundant, but makes some things easier:
    \savedanchor\southwest{%
        \pgfmathsetlength\pgf@x{\pgfshapeminwidth}%
        \pgfmathsetlength\pgf@y{\pgfshapeminheight}%
        \pgf@x=-0.5\pgf@x
        \pgf@y=-0.5\pgf@y
    }
    % Inherit from rectangle
    \inheritanchorborder[from=rectangle]

    % Define same anchor a normal rectangle has
    \anchor{center}{\pgfpointorigin}
    \anchor{north}{\northeast \pgf@x=0pt}
    \anchor{east}{\northeast \pgf@y=0pt}
    \anchor{south}{\southwest \pgf@x=0pt}
    \anchor{west}{\southwest \pgf@y=0pt}
    \anchor{north east}{\northeast}
    \anchor{north west}{\northeast \pgf@x=-\pgf@x}
    \anchor{south west}{\southwest}
    \anchor{south east}{\southwest \pgf@x=-\pgf@x}
    \anchor{text}{
        \pgfpointorigin
        \advance\pgf@x by -.5\wd\pgfnodeparttextbox%
        \advance\pgf@y by -.5\ht\pgfnodeparttextbox%
        \advance\pgf@y by +.5\dp\pgfnodeparttextbox%
    }

    % Define anchors for signal ports
    \anchor{Din}{
        \pgf@process{\northeast}%
        \pgf@x=-1\pgf@x%
        \pgf@y=0.75\pgf@y%
    }
    \anchor{Ld}{
        \pgf@process{\northeast}%
        \pgf@x=-1\pgf@x%
        \pgf@y=0.375\pgf@y%
    }
    \anchor{Shift}{
        \pgf@process{\northeast}%
        \pgf@x=-1\pgf@x%
        \pgf@y=0\pgf@y%
    }
    \anchor{CLK}{
        \pgf@process{\northeast}%
        \pgf@x=-1\pgf@x%
        \pgf@y=-.75\pgf@y%
    }
    \anchor{Dout}{
        \pgf@process{\northeast}%
        \pgf@y=0\pgf@y%
    }

    % Draw the rectangle box and the port labels
    \backgroundpath{
        % Rectangle box
        \pgfpathrectanglecorners{\southwest}{\northeast}
        % Angle (>) for clock input
        \pgf@anchor@srr@CLK
        \pgf@xa=\pgf@x \pgf@ya=\pgf@y
        \pgf@xb=\pgf@x \pgf@yb=\pgf@y
        \pgf@xc=\pgf@x \pgf@yc=\pgf@y
        \pgfmathsetlength\pgf@x{1ex} % size depends on font size
        \advance\pgf@ya by \pgf@x
        \advance\pgf@xb by \pgf@x
        \advance\pgf@yc by -\pgf@x
        \pgfpathmoveto{\pgfpoint{\pgf@xa}{\pgf@ya}}
        \pgfpathlineto{\pgfpoint{\pgf@xb}{\pgf@yb}}
        \pgfpathlineto{\pgfpoint{\pgf@xc}{\pgf@yc}}
        \pgfclosepath

        % Draw port labels
        \begingroup
        \tikzset{port labels} % Use font from this style
        \tikz@textfont
        \tikz@textfont

        \pgf@anchor@srr@Din
        \pgftext[left,base,at={\pgfpoint{\pgf@x}{\pgf@y}},x=\pgfshapeinnerxsep]{\raisebox{-0.75ex}{Din}}

        \pgf@anchor@srr@CLK
        \pgftext[left,base,at={\pgfpoint{\pgf@x}{\pgf@y}},x=\pgfshapeinnerxsep]{\raisebox{-0.75ex}{}}

        \pgf@anchor@srr@Dout
        \pgftext[right,base,at={\pgfpoint{\pgf@x}{\pgf@y}},x=-\pgfshapeinnerxsep]{\raisebox{-0.75ex}{Dout}}

        \pgf@anchor@srr@Ld
        \pgftext[left,base,at={\pgfpoint{\pgf@x}{\pgf@y}},x=\pgfshapeinnerxsep]{\raisebox{-0.75ex}{Ld}}

        \pgf@anchor@srr@Shift
        \pgftext[left,base,at={\pgfpoint{\pgf@x}{\pgf@y}},x=\pgfshapeinnerxsep]{\raisebox{-0.75ex}{Shift}}

        \endgroup
    }
}

%
% Data Flip Flip (DFF) shape
%

\pgfdeclareshape{dff}{
  % The 'minimum width' and 'minimum height' keys, not the content, determine
  % the size
    \savedanchor\northeast{%
        \pgfmathsetlength\pgf@x{\pgfshapeminwidth}%
        \pgfmathsetlength\pgf@y{\pgfshapeminheight}%
        \pgf@x=0.5\pgf@x
        \pgf@y=0.5\pgf@y
    }
  % This is redundant, but makes some things easier:
    \savedanchor\southwest{%
        \pgfmathsetlength\pgf@x{\pgfshapeminwidth}%
        \pgfmathsetlength\pgf@y{\pgfshapeminheight}%
        \pgf@x=-0.5\pgf@x
        \pgf@y=-0.5\pgf@y
    }
    % Inherit from rectangle
    \inheritanchorborder[from=rectangle]

    % Define same anchor a normal rectangle has
    \anchor{center}{\pgfpointorigin}
    \anchor{north}{\northeast \pgf@x=0pt}
    \anchor{east}{\northeast \pgf@y=0pt}
    \anchor{south}{\southwest \pgf@x=0pt}
    \anchor{west}{\southwest \pgf@y=0pt}
    \anchor{north east}{\northeast}
    \anchor{north west}{\northeast \pgf@x=-\pgf@x}
    \anchor{south west}{\southwest}
    \anchor{south east}{\southwest \pgf@x=-\pgf@x}
    \anchor{text}{
        \pgfpointorigin
        \advance\pgf@x by -.5\wd\pgfnodeparttextbox%
        \advance\pgf@y by -.5\ht\pgfnodeparttextbox%
        \advance\pgf@y by +.5\dp\pgfnodeparttextbox%
    }

    % Define anchors for signal ports
    \anchor{D}{
        \pgf@process{\northeast}%
        \pgf@x=-1\pgf@x%
        \pgf@y=.5\pgf@y%
    }
    \anchor{CLK}{
        \pgf@process{\northeast}%
        \pgf@x=-1\pgf@x%
        \pgf@y=-.5\pgf@y%
    }
    \anchor{Q}{
        \pgf@process{\northeast}%
        \pgf@y=.5\pgf@y%
    }
    \anchor{Qn}{
        \pgf@process{\northeast}%
        \pgf@y=-.5\pgf@y%
    }

    \anchor{R}{
        \pgf@process{\northeast}%
        \pgf@x=0pt%
    }
    \anchor{S}{
        \pgf@process{\northeast}%
        \pgf@x=0pt%
        \pgf@y=-\pgf@y%
    }

    % Draw the rectangle box and the port labels
    \backgroundpath{
        % Rectangle box
        \pgfpathrectanglecorners{\southwest}{\northeast}
        % Angle (>) for clock input
        \pgf@anchor@dff@CLK
        \pgf@xa=\pgf@x \pgf@ya=\pgf@y
        \pgf@xb=\pgf@x \pgf@yb=\pgf@y
        \pgf@xc=\pgf@x \pgf@yc=\pgf@y
        \pgfmathsetlength\pgf@x{1ex} % size depends on font size
        \advance\pgf@ya by \pgf@x
        \advance\pgf@xb by \pgf@x
        \advance\pgf@yc by -\pgf@x
        \pgfpathmoveto{\pgfpoint{\pgf@xa}{\pgf@ya}}
        \pgfpathlineto{\pgfpoint{\pgf@xb}{\pgf@yb}}
        \pgfpathlineto{\pgfpoint{\pgf@xc}{\pgf@yc}}
        \pgfclosepath

        % Draw port labels
        \begingroup
        \tikzset{port labels} % Use font from this style
        \tikz@textfont

        \pgf@anchor@dff@D
        \pgftext[left,base,at={\pgfpoint{\pgf@x}{\pgf@y}},x=\pgfshapeinnerxsep]{\raisebox{-0.75ex}{D}}

        \pgf@anchor@dff@Q
        \pgftext[right,base,at={\pgfpoint{\pgf@x}{\pgf@y}},x=-\pgfshapeinnerxsep]{\raisebox{-.75ex}{Q}}

        \pgf@anchor@dff@Qn
        \pgftext[right,base,at={\pgfpoint{\pgf@x}{\pgf@y}},x=-\pgfshapeinnerxsep]{\raisebox{-.75ex}{$\overline{\mbox{Q}}$}}

        \pgf@anchor@dff@R
        \pgftext[top,at={\pgfpoint{\pgf@x}{\pgf@y}},y=-\pgfshapeinnerysep]{R}

        \pgf@anchor@dff@S
        \pgftext[bottom,at={\pgfpoint{\pgf@x}{\pgf@y}},y=\pgfshapeinnerysep]{S}

        \endgroup
    }
}


%
% External syndrome block
%

\pgfdeclareshape{syndrome}{
  % The 'minimum width' and 'minimum height' keys, not the content, determine
  % the size
    \savedanchor\northeast{%
        \pgfmathsetlength\pgf@x{\pgfshapeminwidth}%
        \pgfmathsetlength\pgf@y{\pgfshapeminheight}%
        \pgf@x=0.5\pgf@x
        \pgf@y=0.5\pgf@y
    }
  % This is redundant, but makes some things easier:
    \savedanchor\southwest{%
        \pgfmathsetlength\pgf@x{\pgfshapeminwidth}%
        \pgfmathsetlength\pgf@y{\pgfshapeminheight}%
        \pgf@x=-0.5\pgf@x
        \pgf@y=-0.5\pgf@y
    }
    % Inherit from rectangle
    \inheritanchorborder[from=rectangle]

    % Define same anchor a normal rectangle has
    \anchor{center}{\pgfpointorigin}
    \anchor{north}{\northeast \pgf@x=0pt}
    \anchor{east}{\northeast \pgf@y=0pt}
    \anchor{south}{\southwest \pgf@x=0pt}
    \anchor{west}{\southwest \pgf@y=0pt}
    \anchor{north east}{\northeast}
    \anchor{north west}{\northeast \pgf@x=-\pgf@x}
    \anchor{south west}{\southwest}
    \anchor{south east}{\southwest \pgf@x=-\pgf@x}
    \anchor{text}{
        \pgfpointorigin
        \advance\pgf@x by -.5\wd\pgfnodeparttextbox%
        \advance\pgf@y by -.5\ht\pgfnodeparttextbox%
        \advance\pgf@y by +.5\dp\pgfnodeparttextbox%
    }

    % Define anchors for signal ports
    \anchor{Data}{
        \pgf@process{\northeast}%
        \pgf@x=-1\pgf@x%
        \pgf@y=.75\pgf@y%
    }
    \anchor{Start}{
        \pgf@process{\northeast}%
        \pgf@x=-1\pgf@x%
        \pgf@y=.25\pgf@y%
    }
    \anchor{Reset}{
        \pgf@process{\northeast}%
        \pgf@x=-1\pgf@x%
        \pgf@y=-.25\pgf@y%
    }
    \anchor{CLK}{
        \pgf@process{\northeast}%
        \pgf@x=-1\pgf@x%
        \pgf@y=-.75\pgf@y%
    }

    \anchor{Syndrome}{
        \pgf@process{\northeast}%
        \pgf@y=.5\pgf@y%
    }
    \anchor{End}{
        \pgf@process{\northeast}%
        \pgf@y=-.5\pgf@y%
    }

    % Draw the rectangle box and the port labels
    \backgroundpath{
        % Rectangle box
        \pgfpathrectanglecorners{\southwest}{\northeast}
        % Angle (>) for clock input
        \pgf@anchor@syndrome@CLK
        \pgf@xa=\pgf@x \pgf@ya=\pgf@y
        \pgf@xb=\pgf@x \pgf@yb=\pgf@y
        \pgf@xc=\pgf@x \pgf@yc=\pgf@y
        \pgfmathsetlength\pgf@x{0.5ex} % size depends on font size
        \advance\pgf@ya by \pgf@x
        \advance\pgf@xb by \pgf@x
        \advance\pgf@yc by -\pgf@x
        \pgfpathmoveto{\pgfpoint{\pgf@xa}{\pgf@ya}}
        \pgfpathlineto{\pgfpoint{\pgf@xb}{\pgf@yb}}
        \pgfpathlineto{\pgfpoint{\pgf@xc}{\pgf@yc}}
        \pgfclosepath

        % Draw port labels
        \begingroup
        \tikzset{port labels} % Use font from this style
        \tikz@textfont

        \pgf@anchor@syndrome@Data
        \pgftext[left,base,at={\pgfpoint{\pgf@x}{\pgf@y}},x=\pgfshapeinnerxsep]{\raisebox{-0.75ex}{Data [31:0]}}

        \pgf@anchor@syndrome@Start
        \pgftext[left,base,at={\pgfpoint{\pgf@x}{\pgf@y}},x=\pgfshapeinnerxsep]{\raisebox{-0.75ex}{Start}}

        \pgf@anchor@syndrome@Reset
        \pgftext[left,base,at={\pgfpoint{\pgf@x}{\pgf@y}},x=\pgfshapeinnerxsep]{\raisebox{-0.75ex}{Reset}}

        \pgf@anchor@syndrome@CLK
        \pgftext[left,base,at={\pgfpoint{\pgf@x}{\pgf@y}},x=\pgfshapeinnerxsep]{\raisebox{-0.75ex}{CLK}}

        \pgf@anchor@syndrome@Syndrome
        \pgftext[right,base,at={\pgfpoint{\pgf@x}{\pgf@y}},x=-\pgfshapeinnerxsep]{\raisebox{-.75ex}{Syndrome [9:0]}}

        \pgf@anchor@syndrome@End
        \pgftext[right,base,at={\pgfpoint{\pgf@x}{\pgf@y}},x=-\pgfshapeinnerxsep]{\raisebox{-.75ex}{End}}

        \endgroup
    }
}

%
% External ut_syndrome block
%

\pgfdeclareshape{utsyndrome}{
  % The 'minimum width' and 'minimum height' keys, not the content, determine
  % the size
    \savedanchor\northeast{%
        \pgfmathsetlength\pgf@x{\pgfshapeminwidth}%
        \pgfmathsetlength\pgf@y{\pgfshapeminheight}%
        \pgf@x=0.5\pgf@x
        \pgf@y=0.5\pgf@y
    }
  % This is redundant, but makes some things easier:
    \savedanchor\southwest{%
        \pgfmathsetlength\pgf@x{\pgfshapeminwidth}%
        \pgfmathsetlength\pgf@y{\pgfshapeminheight}%
        \pgf@x=-0.5\pgf@x
        \pgf@y=-0.5\pgf@y
    }
    % Inherit from rectangle
    \inheritanchorborder[from=rectangle]

    % Define same anchor a normal rectangle has
    \anchor{center}{\pgfpointorigin}
    \anchor{north}{\northeast \pgf@x=0pt}
    \anchor{east}{\northeast \pgf@y=0pt}
    \anchor{south}{\southwest \pgf@x=0pt}
    \anchor{west}{\southwest \pgf@y=0pt}
    \anchor{north east}{\northeast}
    \anchor{north west}{\northeast \pgf@x=-\pgf@x}
    \anchor{south west}{\southwest}
    \anchor{south east}{\southwest \pgf@x=-\pgf@x}
    \anchor{text}{
        \pgfpointorigin
        \advance\pgf@x by -.5\wd\pgfnodeparttextbox%
        \advance\pgf@y by -.5\ht\pgfnodeparttextbox%
        \advance\pgf@y by +.5\dp\pgfnodeparttextbox%
    }

    % Define anchors for signal ports

    \anchor{Data}{
        \pgf@process{\northeast}%
        \pgf@x=-1\pgf@x%
        \pgf@y=.8\pgf@y%
    }

    \anchor{ld}{
        \pgf@process{\northeast}%
        \pgf@x=-1\pgf@x%
        \pgf@y=.4\pgf@y%
    }

    \anchor{calc}{
        \pgf@process{\northeast}%
        \pgf@x=-1\pgf@x%
        \pgf@y=0\pgf@y%
    }

    \anchor{clear}{
        \pgf@process{\northeast}%
        \pgf@x=-1\pgf@x%
        \pgf@y=-0.4\pgf@y%
    }

    \anchor{CLK}{
        \pgf@process{\northeast}%
        \pgf@x=-1\pgf@x%
        \pgf@y=-.8\pgf@y%
    }

    \anchor{Syndrome}{
        \pgf@process{\northeast}%
        \pgf@y=0.5\pgf@y%
    }

    % Draw the rectangle box and the port labels
    \backgroundpath{
        % Rectangle box
        \pgfpathrectanglecorners{\southwest}{\northeast}
        % Angle (>) for clock input
        \pgf@anchor@utsyndrome@CLK
        \pgf@xa=\pgf@x \pgf@ya=\pgf@y
        \pgf@xb=\pgf@x \pgf@yb=\pgf@y
        \pgf@xc=\pgf@x \pgf@yc=\pgf@y
        \pgfmathsetlength\pgf@x{0.5ex} % size depends on font size
        \advance\pgf@ya by \pgf@x
        \advance\pgf@xb by \pgf@x
        \advance\pgf@yc by -\pgf@x
        \pgfpathmoveto{\pgfpoint{\pgf@xa}{\pgf@ya}}
        \pgfpathlineto{\pgfpoint{\pgf@xb}{\pgf@yb}}
        \pgfpathlineto{\pgfpoint{\pgf@xc}{\pgf@yc}}
        \pgfclosepath

        % Draw port labels
        \begingroup
        \tikzset{port labels} % Use font from this style
        \tikz@textfont

        \pgf@anchor@utsyndrome@Data
        \pgftext[left,base,at={\pgfpoint{\pgf@x}{\pgf@y}},x=\pgfshapeinnerxsep]{\raisebox{-0.75ex}{Data [31:0]}}

        \pgf@anchor@utsyndrome@ld
        \pgftext[left,base,at={\pgfpoint{\pgf@x}{\pgf@y}},x=\pgfshapeinnerxsep]{\raisebox{-0.75ex}{Ld}}

        \pgf@anchor@utsyndrome@calc
        \pgftext[left,base,at={\pgfpoint{\pgf@x}{\pgf@y}},x=\pgfshapeinnerxsep]{\raisebox{-0.75ex}{calc}}

        \pgf@anchor@utsyndrome@clear
        \pgftext[left,base,at={\pgfpoint{\pgf@x}{\pgf@y}},x=\pgfshapeinnerxsep]{\raisebox{-0.75ex}{clear}}

        \pgf@anchor@utsyndrome@CLK
        \pgftext[left,base,at={\pgfpoint{\pgf@x}{\pgf@y}},x=\pgfshapeinnerxsep]{\raisebox{-0.75ex}{CLK}}

        \pgf@anchor@utsyndrome@Syndrome
        \pgftext[right,base,at={\pgfpoint{\pgf@x}{\pgf@y}},x=-\pgfshapeinnerxsep]{\raisebox{-.75ex}{Syndrome [9:0]}}

        \endgroup
    }
}

%
% External uc_syndrome block
%

\pgfdeclareshape{ucsyndrome}{
  % The 'minimum width' and 'minimum height' keys, not the content, determine
  % the size
    \savedanchor\northeast{%
        \pgfmathsetlength\pgf@x{\pgfshapeminwidth}%
        \pgfmathsetlength\pgf@y{\pgfshapeminheight}%
        \pgf@x=0.5\pgf@x
        \pgf@y=0.5\pgf@y
    }
  % This is redundant, but makes some things easier:
    \savedanchor\southwest{%
        \pgfmathsetlength\pgf@x{\pgfshapeminwidth}%
        \pgfmathsetlength\pgf@y{\pgfshapeminheight}%
        \pgf@x=-0.5\pgf@x
        \pgf@y=-0.5\pgf@y
    }
    % Inherit from rectangle
    \inheritanchorborder[from=rectangle]

    % Define same anchor a normal rectangle has
    \anchor{center}{\pgfpointorigin}
    \anchor{north}{\northeast \pgf@x=0pt}
    \anchor{east}{\northeast \pgf@y=0pt}
    \anchor{south}{\southwest \pgf@x=0pt}
    \anchor{west}{\southwest \pgf@y=0pt}
    \anchor{north east}{\northeast}
    \anchor{north west}{\northeast \pgf@x=-\pgf@x}
    \anchor{south west}{\southwest}
    \anchor{south east}{\southwest \pgf@x=-\pgf@x}
    \anchor{text}{
        \pgfpointorigin
        \advance\pgf@x by -.5\wd\pgfnodeparttextbox%
        \advance\pgf@y by -.5\ht\pgfnodeparttextbox%
        \advance\pgf@y by +.5\dp\pgfnodeparttextbox%
    }

    % Define anchors for signal ports

    \anchor{Start}{
        \pgf@process{\northeast}%
        \pgf@x=-1\pgf@x%
        \pgf@y=0.5\pgf@y%
    }

    \anchor{Reset}{
        \pgf@process{\northeast}%
        \pgf@x=-1\pgf@x%
        \pgf@y=0\pgf@y%
    }

    \anchor{CLK}{
        \pgf@process{\northeast}%
        \pgf@x=-1\pgf@x%
        \pgf@y=-.75\pgf@y%
    }

    \anchor{End}{
        \pgf@process{\northeast}%
        \pgf@y=0.75\pgf@y%
    }

    \anchor{Calc}{
        \pgf@process{\northeast}%
        \pgf@y=-0.25\pgf@y%
    }

    \anchor{Ld}{
        \pgf@process{\northeast}%
        \pgf@y=0.25\pgf@y%
    }

    \anchor{Clear}{
        \pgf@process{\northeast}%
        \pgf@y=-0.75\pgf@y%
    }

    % Draw the rectangle box and the port labels
    \backgroundpath{
        % Rectangle box
        \pgfpathrectanglecorners{\southwest}{\northeast}
        % Angle (>) for clock input
        \pgf@anchor@ucsyndrome@CLK
        \pgf@xa=\pgf@x \pgf@ya=\pgf@y
        \pgf@xb=\pgf@x \pgf@yb=\pgf@y
        \pgf@xc=\pgf@x \pgf@yc=\pgf@y
        \pgfmathsetlength\pgf@x{0.5ex} % size depends on font size
        \advance\pgf@ya by \pgf@x
        \advance\pgf@xb by \pgf@x
        \advance\pgf@yc by -\pgf@x
        \pgfpathmoveto{\pgfpoint{\pgf@xa}{\pgf@ya}}
        \pgfpathlineto{\pgfpoint{\pgf@xb}{\pgf@yb}}
        \pgfpathlineto{\pgfpoint{\pgf@xc}{\pgf@yc}}
        \pgfclosepath

        % Draw port labels
        \begingroup
        \tikzset{port labels} % Use font from this style
        \tikz@textfont

        \pgf@anchor@ucsyndrome@Start
        \pgftext[left,base,at={\pgfpoint{\pgf@x}{\pgf@y}},x=\pgfshapeinnerxsep]{\raisebox{-0.75ex}{Start}}

        \pgf@anchor@ucsyndrome@Reset
        \pgftext[left,base,at={\pgfpoint{\pgf@x}{\pgf@y}},x=\pgfshapeinnerxsep]{\raisebox{-0.75ex}{Reset}}

        \pgf@anchor@ucsyndrome@CLK
        \pgftext[left,base,at={\pgfpoint{\pgf@x}{\pgf@y}},x=\pgfshapeinnerxsep]{\raisebox{-0.75ex}{CLK}}

        \pgf@anchor@ucsyndrome@End
        \pgftext[right,base,at={\pgfpoint{\pgf@x}{\pgf@y}},x=-\pgfshapeinnerxsep]{\raisebox{-.75ex}{End}}

        \pgf@anchor@ucsyndrome@Calc
        \pgftext[right,base,at={\pgfpoint{\pgf@x}{\pgf@y}},x=-\pgfshapeinnerxsep]{\raisebox{-.75ex}{Calc}}

        \pgf@anchor@ucsyndrome@Ld
        \pgftext[right,base,at={\pgfpoint{\pgf@x}{\pgf@y}},x=-\pgfshapeinnerxsep]{\raisebox{-.75ex}{Ld}}

        \pgf@anchor@ucsyndrome@Clear
        \pgftext[right,base,at={\pgfpoint{\pgf@x}{\pgf@y}},x=-\pgfshapeinnerxsep]{\raisebox{-.75ex}{Clear}}

        \endgroup
    }
}

\makeatother

%
% Glossaries
%

\makeglossaries

%
% Document
%

\title{Décodeur BCH}
\author{Ronan~\bsc{Le Guillou} \and Arnaud~\bsc{Levaufre} \and Bastien~\bsc{Orivel}}

\begin{document}
    \maketitle
    \tableofcontents
    \printglossaries
    \listoffigures
    \listoftables

    \chapter{Introduction}

    Ce rapport présente le travail de conception et d'implémentation d'un
    décodeur BCH réalisé dans le cadre du mini projet du cours de conception
    sur puce de l'Enib.

    \paragraph{}
    Le décodeur BCH demandé doit être capable de détecter jusqu'a cinq erreurs
    et d'en corriger au plus deux. Le décodeur travail sur des données codées
    sur 21 bits et utilise 10 bits de redodance. La pair données et bits de
    redondance forment un mot de 31 bits auquel on ajoute un bit 0 en poid for
    pour former un mot de 32 bits. Le décodeur doit implémenter une interface
    avalon le rendant compatible avec le processeur NiosII. Les communications
    entre le décodeur et le processeur se fait par quatres registres: les
    registres de status, de controle de donnée d'entrée et de sortie. Ces
    registres ont un format de 32 bits.

    \paragraph{}
    Pour la lecture de ce rapport, il est important de noter plusieurs points
    sur les schémas fonctionnels présentés. La connexion entre deux cables
    croisés est représenté par un points. Deux cables croisés sans points ne se
    connectent pas. Les entrées d'horloge et de reset sont présents sur les
    blocks qui en ont besoins mais ne sont pas cablés sur les schéma pour
    faciliter la lecture. Toute entrée reset et Clk est par défaut et sauf
    indication contraire, considéré comme reliée à l'horloge et au reset
    maitre. Les cas particuliers sont explicités.

    \chapter{Étude du circuit}
        \section{Architecture globale}
        La vue extérieur du décodeur BCH est imposé par le sujet et est
        présentée dans le figure~\ref{fig:bch} ci-dessous. Sa structure
        interne est présentée dans la subsection~\ref{sec:fonc_bch}. On
        retrouve les entrées et sorties nécéssaire pour définir une
        interface Avalon fonctionnelle.

            \begin{figure}[H]
                \centering
                \begin{tikzpicture}[>=stealth,scale=1, every node/.style={scale=1}, circuit logic US]
                    \tikzstyle{every initial by arrow} = [initial text=reset, text=red, -, draw=red, decorate, decoration=zigzag]
                    \node[shape=bch] (bch) {BCH};

                    \draw[<-, double] (bch.Din) -- +(-1, 0) node[anchor=east] {Din};
                    \draw[<-] (bch.Wr) -- +(-1, 0) node[anchor=east] {Wr};
                    \draw[<-] (bch.Rd) -- +(-1, 0) node[anchor=east] {Rd};
                    \draw[<-] (bch.Addr) -- +(-1, 0) node[anchor=east] {Addr};
                    \draw[<-] (bch.Resetn) -- +(-1, 0) node[anchor=east] {Reset\_n};
                    \draw[->, double] (bch.Dout) -- +(1, 0) node[anchor=west] {Dout};
                    \draw[->] (bch.Irqn) -- +(1, 0) node[anchor=west] {Irq\_n};

                \end{tikzpicture}
                \caption{Vue éxtérieur du block BCH}
                \label{fig:bch}
            \end{figure}

            \subsection{Sous ensembldes fonctionnels du block BCH}
            \label{sec:fonc_bch}
            La réalisation du décodeur BCH se base sur cinq blocks
            présentés dans les subsection suivantes.
            \begin{itemize}
                \item Le calcul du syndrome.
                \item La look up table.
                \item Le correcteur d'erreur.
                \item L'interface avalon.
                \item La machine a état maitre.
            \end{itemize}
            A l'eception de l'interface avaalon et de la machine a état, les
            blocks sont relativement indépendants et peuvent être utilisé
            chacun séparément. Cela facilite, d'une part, le développement et
            le test et, d'autre part, la réutilisation du code développé. Un
            codeur BCH pourrait par exemple réutiliser le block de calcul du
            syndrome. La figure~\ref{fig:bch_blocks}

            \begin{figure}[H]
                \centering
                \begin{tikzpicture}[>=stealth',shorten >=1pt,auto,node distance=4cm, scale=0.75, every node/.style={scale=0.75}]
                    \tikzstyle{every initial by arrow} = [initial text=reset, text=red, -, draw=red, decorate, decoration=zigzag]
                    \tikzstyle{connectionpoint} = [circle, draw, fill=black, scale=0.5];
                    %\draw[help lines] (0, 0) grid (20,20);

                    \node[shape=avalon] at (0, 0) (avalon) {Avalon};
                    \node[shape=memaster] at (7.25, 10) (memaster) {ME\_Master};
                    \node[shape=syndrome] at (15, 15) (syndrome) {Syndrome};
                    \node[shape=lut] at (15, 10) (lut) {Lut};
                    \node[shape=corr] at (15, 5) (corr) {Correcteur};
                    \node[shape=dff] at (8, 5) (dff) {Err};

                    \draw[->] (memaster.StartSyndrome) -- +(0.25, 0) |- (syndrome.Start);
                    \draw[->] (memaster.StartLut) -- +(0.25, 0) |- (lut.Start);
                    \draw[->] (memaster.StartCorr) -- +(1.25, 0) |- (corr.Start);
                    \draw[->] (corr.End) -- +(1.5, 0) -- +(1.5, 13) -- +(-15, 13) |- (memaster.EndCorr);
                    \draw[->] (lut.End) -- +(1.25, 0) -- +(1.25, 7.25) -- +(-14.75, 7.25) |- (memaster.EndLut);
                    \draw[->] (syndrome.End) -- +(1, 0) -- +(1, 3.75) -- +(-14.5, 3.75) |- (memaster.EndSyndrome);
                    \draw[->, double] (syndrome.Syndrome) -- +(0.75, 0) -- +(0.75, 2) -- +(-14.25, 2) |- (memaster.Syndrome);
                    \draw[->, double] (syndrome.Syndrome) -- +(0.75, 0) node[connectionpoint] {} -- +(0.75, -3) -- +(-5.5, -3) |- (lut.Syndrome);
                    \draw[->, double] (lut.Err) -- +(0.5, 0) |- +(-11, -1.75) |- (dff.D);
                    \draw[->, double] (dff.Q) -- +(1,0) |- (corr.Err);
                    \draw[->] (lut.LdErr) -- +(0.25, 0) |- +(-11.5, -0.75) -- +(-11.5, -5.5) -| (dff.S);
                    \draw[->] (memaster.RazErr) -| +(0.25, -0.5) -| (dff.R);
                    \draw[->] (lut.Pone) -| +(1, -3.75) -- +(-5.5, -3.75) |- (corr.Pone);
                    \draw[->] (lut.Ptwo) -| +(0.75, -2.75) -- +(-5.75, -2.75) |- (corr.Ptwo);

                    \draw[->] (memaster.AskIrq) -- +(0.5, 0) |- (avalon.AskIrq);
                    \draw[->, double] (corr.Dout) -- +(1, 0) |- (avalon.CorrOut);
                    \draw[->] (corr.End) -- +(1.5, 0) node[connectionpoint] {} |- (avalon.CorrOutLd);

                    \draw[->, double] (avalon.FifoOut) -- +(9, 0) |- (corr.Din);
                    \draw[->, double] (avalon.FifoOut) -- +(1, 0) node[connectionpoint] {} -| +(1, 4) -| +(-2, 4) |- (syndrome.Data);
                    \draw[->, double] (avalon.Words) -| +(1.5, 5.25) -| +(-1.5, 5.25) |- (memaster.Words);
                    \draw[->] (avalon.Decode) -| +(2, 6.5) -| +(-1, 6.5) |- (memaster.Decode);

                    \draw (avalon.FifoOut) +(0,0) node [anchor=south west]  {32};
                    \draw (avalon.Words) +(0,0) node [anchor=south west] {5};
                    \draw (dff.Q) +(0,0) node [anchor=south west] {2};
                    \draw (corr.Dout) +(0,0) node [anchor=south west] {32};
                    \draw (lut.Err) +(0,0) node [anchor=south west] {2};
                    \draw (syndrome.Syndrome) +(0,0) node [anchor=south west] {10};

                \end{tikzpicture}
                \caption{Vue intérieure du block BCH}
                \label{fig:bch_blocks}
            \end{figure}

            \subsection{Machine à état principale}
            \begin{figure}[H]
                \centering
                \begin{tikzpicture}[>=stealth',shorten >=1pt,auto,node distance=4cm, scale=0.75, every node/.style={scale=0.75}]
                    \tikzstyle{every initial by arrow} = [initial text=reset, text=red, -, draw=red, decorate, decoration=zigzag]
                    %\draw[help lines] (0, 0) grid (10,10);

                    \node[state, accepting, initial above ] at (5, 10) (E1){Iddle};
                    \node[state] at (10, 5) (E2){Syndrome};
                    \node[state] at (5, 0) (E3){LUT};
                    \node[state] at (0, 5) (E4){Correction};

                    \path[->] (E1) edge [loop below] node {!decode} (E1);
                    \path[->] (E1) edge [bend left] node {decode / start\_syndrome} (E2);

                    \path[->] (E2) edge [loop right] node {!end\_syndrome} (E2);
                    \path[->] (E2) edge [bend left] node {end\_syndrome . syndrome != 0 / start\_lut} (E3);
                    \path[->] (E2) edge [bend left] node {end\_syndrome . syndrome = 0 / raz\_err, start\_corr} (E4);

                    \path[->] (E3) edge [loop below] node {!end\_lut} (E3);
                    \path[->] (E3) edge [bend left] node {end\_lut / start\_corr} (E4);

                    \path[->] (E4) edge [loop left] node {!end\_corr} (E4);
                    \path[->] (E4) edge [bend left] node {end\_corr . words = 0 / ask\_irq} (E1);
                    \path[->] (E4) edge [bend left] node {end\_corr . words > 0} (E2);


                \end{tikzpicture}
                \caption{Machine à état globale}
                \label{fig:me_syndrome}
            \end{figure}


        \section{Calcul du syndrome}
            Le calcul du syndrome est décomposé en une unité de traitement et
            un unité de contrôle qui sont présentées dans les
            sections~\ref{sec:ut_syndrome} et \ref{sec:uc_syndrome}. La
            figure~\ref{fig:syndrome} présenté ci-dessous montre les signaux
            entrants et sortants nécessaire au calcul du syndrome par les
            unités proposées. Le principe de fonctionnement est simple,
            l'utilisateur de ce bloc propose une donnée codée sur 32 bits sur
            l'entrée «~Data~» et démarre le calcul en imposant l'entrée
            «~Start~» à 1.  Une fois calculé le syndrome est disponible sur la
            sortie «~Syndrome~» et le drapeau «~End~» est passé à 1 sur une durée
            d'une période d'horloge.

            \begin{figure}[H]
                \centering
                \begin{tikzpicture}[>=stealth,scale=1, every node/.style={scale=1}, circuit logic US]
                    \node[shape=syndrome] (syndrome) {Syndrome};

                    \draw[<-, double] (syndrome.Data) -- +(-1, 0) node [anchor=east] {};
                    \draw[<-] (syndrome.Start) -- +(-1, 0) node [anchor=east] {};
                    \draw[<-] (syndrome.Reset) -- +(-1, 0) node [anchor=east] {};
                    \draw[<-] (syndrome.CLK) -- +(-1, 0) node [anchor=east] {};

                    \draw[->, double] (syndrome.Syndrome) -- +(1, 0) node [anchor=west] {};
                    \draw[->] (syndrome.End) -- +(1, 0) node [anchor=west] {};
                \end{tikzpicture}
                \caption{Vue extérieur du bloc de calcul du syndrome}
                \label{fig:syndrome}
            \end{figure}

            \subsection{Unité de traitement}
            \label{sec:ut_syndrome}

            L'unité de traitement est présentée en
            figure~\ref{fig:ut_syndrome_ext} et \ref{fig:ut_syndrome} et
            reprend le schéma du sujet en lui ajoutant le registre à décalage
            permettant de rentrer la donnée utilisateur bit par bit, poids
            faible en premier.

            \begin{figure}[H]
                \centering
                \begin{tikzpicture}[>=stealth,scale=1, every node/.style={scale=1}, circuit logic US]
                    \node[shape=utsyndrome] (utsyndrome) {UT Syndrome};

                    \draw[<-, double] (utsyndrome.Data) -- +(-1, 0) node [anchor=east] {};
                    \draw[<-] (utsyndrome.ld) -- +(-1, 0) node [anchor=east] {};
                    \draw[<-] (utsyndrome.calc) -- +(-1, 0) node [anchor=east] {};
                    \draw[<-] (utsyndrome.clear) -- +(-1, 0) node [anchor=east] {};
                    \draw[<-] (utsyndrome.CLK) -- +(-1, 0) node [anchor=east] {};

                    \draw[->, double] (utsyndrome.Syndrome) -- +(1, 0) node [anchor=west] {};

                \end{tikzpicture}
                \caption{vue éxtérieur de l'unité de traitement}
                \label{fig:ut_syndrome_ext}
            \end{figure}

            \begin{figure}[H]
                \centering
                \begin{tikzpicture}[>=stealth,scale=0.375, every node/.style={scale=0.5}, circuit logic US]
                    %\draw[help lines] (0, 0) grid (29, 32);

                    \node[shape=dff] at (12, 30) (dff0) {0};
                    \node[shape=dff] at (12, 27) (dff1) {1};
                    \node[shape=dff] at (12, 24) (dff2) {2};
                    \node[shape=dff] at (12, 21) (dff3) {3};
                    \node[shape=dff] at (12, 18) (dff4) {4};
                    \node[shape=dff] at (12, 15) (dff5) {5};
                    \node[shape=dff] at (12, 12) (dff6) {6};
                    \node[shape=dff] at (12, 9) (dff7) {7};
                    \node[shape=dff] at (12, 6) (dff8) {8};
                    \node[shape=dff] at (12, 3) (dff9) {9};

                    \node[circle,draw,fill=black,scale=0.3] at (9,24)  {};
                    \node[circle,draw,fill=black,scale=0.3] at (9, 18)  {};
                    \node[circle,draw,fill=black,scale=0.3] at (9, 15)  {};
                    \node[circle,draw,fill=black,scale=0.3] at (9, 9)  {};
                    \node[circle,draw,fill=black,scale=0.3] at (9, 6)  {};
                    \node[circle,draw,fill=black,scale=0.3] at (10.5, 28.7)  {};
                    \node[circle,draw,fill=black,scale=0.3] at (10.5, 28.7)  {};
                    \node[circle,draw,fill=black,scale=0.3] at (10.5, 25.7)  {};
                    \node[circle,draw,fill=black,scale=0.3] at (10.5, 22.7)  {};
                    \node[circle,draw,fill=black,scale=0.3] at (10.5, 19.7)  {};
                    \node[circle,draw,fill=black,scale=0.3] at (10.5, 16.7)  {};
                    \node[circle,draw,fill=black,scale=0.3] at (10.5, 13.7)  {};
                    \node[circle,draw,fill=black,scale=0.3] at (10.5, 10.7)  {};
                    \node[circle,draw,fill=black,scale=0.3] at (10.5, 7.7)   {};
                    \node[circle,draw,fill=black,scale=0.3] at (10.5, 4.7)   {};
                    \node[circle,draw,fill=black,scale=0.3] at (10.5, 1.7)   {};
                    \node[circle,draw,fill=black,scale=0.3] at (10, 31.3)   {};
                    \node[circle,draw,fill=black,scale=0.3] at (10, 28.3)   {};
                    \node[circle,draw,fill=black,scale=0.3] at (10, 25.3)   {};
                    \node[circle,draw,fill=black,scale=0.3] at (10, 22.3)   {};
                    \node[circle,draw,fill=black,scale=0.3] at (10, 19.3)   {};
                    \node[circle,draw,fill=black,scale=0.3] at (10, 16.3)   {};
                    \node[circle,draw,fill=black,scale=0.3] at (10, 13.3)   {};
                    \node[circle,draw,fill=black,scale=0.3] at (10, 10.3)   {};
                    \node[circle,draw,fill=black,scale=0.3] at (10, 7.3)    {};
                    \node[circle,draw,fill=black,scale=0.3] at (10, 4.3)    {};

                    \node[circle,draw,fill=black,scale=0.3] at (13, 30.5)    {};
                    \node[circle,draw,fill=black,scale=0.3] at (13, 27.5)    {};
                    \node[circle,draw,fill=black,scale=0.3] at (13, 24.5)    {};
                    \node[circle,draw,fill=black,scale=0.3] at (13, 21.5)    {};
                    \node[circle,draw,fill=black,scale=0.3] at (13, 18.5)    {};
                    \node[circle,draw,fill=black,scale=0.3] at (13, 15.5)    {};
                    \node[circle,draw,fill=black,scale=0.3] at (13, 12.5)    {};
                    \node[circle,draw,fill=black,scale=0.3] at (13, 9.5)     {};
                    \node[circle,draw,fill=black,scale=0.3] at (13, 6.5)     {};
                    \node[circle,draw,fill=black,scale=0.3] at (13, 3.5)     {};

                    \node[xor gate, rotate=-90, scale=1.5] at (9.5, 22) (x3) {};
                    \node[xor gate, rotate=-90, scale=1.5] at (9.5, 16) (x5) {};
                    \node[xor gate, rotate=-90, scale=1.5] at (9.5, 13) (x6) {};
                    \node[xor gate, rotate=-90, scale=1.5] at (9.5, 7) (x8) {};
                    \node[xor gate, rotate=-90, scale=1.5] at (9.5, 4) (x9) {};
                    \node[xor gate, rotate=90, scale=1.5] at (8, 28) (x10) {};

                    \node[shape=srr] at (5, 24) (synbuff) {};

                    \node[left] at (0, 26) (data) {Data $[~31:0~]$};
                    \node[left] at (0, 25) (ld) {Ld};
                    \node[left] at (0, 24) (calc) {Calc};
                    \node[right] at (19, 31.5) (syndrome) {Syndrome $[~9:0~]$};
                    \node[left] at (0, 23) (clear) {Clear} ;

                    \draw [->, double] (data) -- (2.5, 26) |- (synbuff.Din);
                    \draw [->] (ld) -- (2, 25) |- (synbuff.Ld);

                    \draw [->] (synbuff.Dout) -| (x10.input 1);
                    \draw [->] (x10.output) -- (8, 29) -- (9, 29) -- (9, 24) -| (x3.input 2);
                    \draw [->] (x10.output) -- (8, 29) -- (9, 29) -- (9, 24) -- (9, 18) -| (x5.input 2);
                    \draw [->] (x10.output) -- (8, 29) -- (9, 29) -- (9, 24) -- (9, 15) -| (x6.input 2);
                    \draw [->] (x10.output) -- (8, 29) -- (9, 29) -- (9, 24) -- (9, 9) -| (x8.input 2);
                    \draw [->] (x10.output) -- (8, 29) -- (9, 29) -- (9, 24) -- (9, 6) -| (x9.input 2);

                    \draw [->] (x10.output) |- (dff0.D);
                    \draw [->] (x9.output) |- (dff9.D);
                    \draw [->] (x8.output) |- (dff8.D);
                    \draw [->] (x6.output) |- (dff6.D);
                    \draw [->] (x5.output) |- (dff5.D);
                    \draw [->] (x3.output) |- (dff3.D);

                    \draw [->] (dff0.Q) -| (13, 28.5) -- (11, 28.5) |- (dff1.D);
                    \draw [->] (dff1.Q) -| (13, 25.5) -- (11, 25.5) |- (dff2.D);
                    \draw [->] (dff2.Q) -| (13, 22.5) -- (13, 22.5) -| (x3.input 1);
                    \draw [->] (dff3.Q) -| (13, 19.5) -- (11, 19.5) |- (dff4.D);
                    \draw [->] (dff4.Q) -| (13, 16.5) -- (13, 16.5) -| (x5.input 1);
                    \draw [->] (dff5.Q) -| (13, 13.5) -- (13, 13.5) -| (x6.input 1);
                    \draw [->] (dff6.Q) -| (13, 10.5) -- (11, 10.5) |- (dff7.D);
                    \draw [->] (dff7.Q) -| (13, 7.5) -- (13, 7.5) -| (x8.input 1);
                    \draw [->] (dff8.Q) -| (13, 4.5) -- (13, 4.5) -| (x9.input 1);
                    \draw [->] (dff9.Q) -| (13, 1.5) -- (11, 1.5) -| (x10.input 2);

                    \draw [->] (calc) -- (2, 24) -- (2, 21) -- (10.5, 21) -- (10.5, 28.7) -| (dff0.S);
                    \draw [->] (calc) -- (2, 24) -- (2, 21) -- (10.5, 21) -- (10.5, 25.7) -| (dff1.S);
                    \draw [->] (calc) -- (2, 24) -- (2, 21) -- (10.5, 21) -- (10.5, 22.7) -| (dff2.S);
                    \draw [->] (calc) -- (2, 24) -- (2, 21) -- (10.5, 21) -- (10.5, 19.7) -| (dff3.S);
                    \draw [->] (calc) -- (2, 24) -- (2, 21) -- (10.5, 21) -- (10.5, 16.7) -| (dff4.S);
                    \draw [->] (calc) -- (2, 24) -- (2, 21) -- (10.5, 21) -- (10.5, 13.7) -| (dff5.S);
                    \draw [->] (calc) -- (2, 24) -- (2, 21) -- (10.5, 21) -- (10.5, 10.7) -| (dff6.S);
                    \draw [->] (calc) -- (2, 24) -- (2, 21) -- (10.5, 21) -- (10.5, 7.7) -| (dff7.S);
                    \draw [->] (calc) -- (2, 24) -- (2, 21) -- (10.5, 21) -- (10.5, 4.7) -| (dff8.S);
                    \draw [->] (calc) -- (2, 24) -- (2, 21) -- (10.5, 21) -- (10.5, 1.7) -| (dff9.S);

                    \draw [->] (clear) -- (1, 23) -- (1, 20) -- (10, 20) -- (10, 31.3) -| (dff0.R);
                    \draw [->] (clear) -- (1, 23) -- (1, 20) -- (10, 20) -- (10, 28.3) -| (dff1.R);
                    \draw [->] (clear) -- (1, 23) -- (1, 20) -- (10, 20) -- (10, 25.3) -| (dff2.R);
                    \draw [->] (clear) -- (1, 23) -- (1, 20) -- (10, 20) -- (10, 22.3) -| (dff3.R);
                    \draw [->] (clear) -- (1, 23) -- (1, 20) -- (10, 20) -- (10, 19.3) -| (dff4.R);
                    \draw [->] (clear) -- (1, 23) -- (1, 20) -- (10, 20) -- (10, 16.3) -| (dff5.R);
                    \draw [->] (clear) -- (1, 23) -- (1, 20) -- (10, 20) -- (10, 13.3) -| (dff6.R);
                    \draw [->] (clear) -- (1, 23) -- (1, 20) -- (10, 20) -- (10, 10.3) -| (dff7.R);
                    \draw [->] (clear) -- (1, 23) -- (1, 20) -- (10, 20) -- (10, 7.3) -| (dff8.R);
                    \draw [->] (clear) -- (1, 23) -- (1, 20) -- (10, 20) -- (10, 4.3) -| (dff9.R);

                    \draw [->] (dff0.Q) -- (18, 30.5) node [above left](b0) {0};
                    \draw [->] (dff1.Q) -- (18, 27.5) node [above left](b1) {1};
                    \draw [->] (dff2.Q) -- (18, 24.5) node [above left](b2) {2};
                    \draw [->] (dff3.Q) -- (18, 21.5) node [above left](b3) {3};
                    \draw [->] (dff4.Q) -- (18, 18.5) node [above left](b4) {4};
                    \draw [->] (dff5.Q) -- (18, 15.5) node [above left](b5) {5};
                    \draw [->] (dff6.Q) -- (18, 12.5) node [above left](b6) {6};
                    \draw [->] (dff7.Q) -- (18, 9.5) node [above left](b7) {7};
                    \draw [->] (dff8.Q) -- (18, 6.5) node [above left](b8) {8};
                    \draw [->] (dff9.Q) -- (18, 3.5) node [above left](b9) {9};
                    \draw [->, double] (18, 3.5) |- (syndrome);

                \end{tikzpicture}
                \caption{Unité de traitement pour le calcul du syndrome}
                \label{fig:ut_syndrome}
            \end{figure}

            \subsection{Unité de contrôle}
            \label{sec:uc_syndrome}

            L'unité de controle utilisé pour le calcul du syndrome est définie
            par la machine à étât présenté en figure~\ref{fig:uc_syndrome_ext}
            et \ref{fig:me_syndrome}.  Elle se base sur un compteur afin
            d'autoriser le décalage et calcul du syndrome pendant les 31 coups
            d'horloge nécéssaire. Après les 31 coups d'horloge le résultat est
            conservé dans les registres permettants le calcul du syndrome. Il
            sera conservé jusqua la prochaine demande de calcul de syndrome,
            qui débutera par le chargement du registre à décalage avec les
            données issues de la pile et par la remise à zéro des registres de
            calculs du syndrome.

            \begin{figure}[H]
                \centering
                \begin{tikzpicture}[>=stealth,scale=1, every node/.style={scale=1}, circuit logic US]
                    \node[shape=ucsyndrome] (ucsyndrome) {UC Syndrome};

                    \draw[<-] (ucsyndrome.Start) -- +(-1, 0) node [anchor=east] {};
                    \draw[<-] (ucsyndrome.Reset) -- +(-1, 0) node [anchor=east] {};
                    \draw[<-] (ucsyndrome.CLK) -- +(-1, 0) node [anchor=east] {};

                    \draw[->] (ucsyndrome.End) -- +(1, 0) node [anchor=west] {};
                    \draw[->] (ucsyndrome.Calc) -- +(1, 0) node [anchor=west] {};
                    \draw[->] (ucsyndrome.Ld) -- +(1, 0) node [anchor=west] {};
                    \draw[->] (ucsyndrome.Clear) -- +(1, 0) node [anchor=west] {};

                \end{tikzpicture}
                \caption{Vue éxtérieur de l'unité de controle du calcul du syndrome}
                \label{fig:uc_syndrome_ext}
            \end{figure}

            \begin{figure}[H]
                \begin{tikzpicture}[>=stealth',shorten >=1pt,auto,node distance=4cm]
                    \tikzstyle{every initial by arrow} = [text=red, -, draw=red, decorate, decoration=zigzag]
                    \node[state,accepting, initial, initial text=reset](E1){Repos};
                    \node[state](E2)[below of=E1]{Dec\_syn};

                    \path[->] (E1) edge [loop right] node {!start\_syndrome} (E1);
                    \path[->] (E1) edge node {start\_syndrome / ld\_syn\_buff, clear, syn\_dcpt=30} (E2);
                    \path[->] (E2) edge [loop right] node {syn\_dcpt != 0 / calc, syn\_dcpt -= 1} (E2);
                    \path[->] (E2) edge [bend left] node {syn\_dcpt == 0 / end\_syndrome} (E1);

                \end{tikzpicture}
                \caption{Machine à état utilisée pour le calcul du syndrome}
                \label{fig:me_syndrome}
            \end{figure}

            \subsection{Définition des connexions entre l'UC et l'UT}

            \begin{figure}[H]
                \centering
                \begin{tikzpicture}[>=stealth, node distance=3cm, scale=0.8, every node/.style={scale=0.8}, circuit logic US]
                    \node[shape=ucsyndrome] at (5, 5) (ucsyndrome) {UC Syndrome};
                    \node[shape=utsyndrome] (utsyndrome) [below of=ucsyndrome] {UT Syndrome};

                    \draw[<-] (ucsyndrome.Start) -- +(-3, 0) node [anchor=east] {start\_syndrome};
                    \draw[<-] (ucsyndrome.Reset) -- +(-3, 0) node [anchor=east] {reset};
                    \draw[<-] (ucsyndrome.CLK) -- +(-3, 0) node [anchor=east] (CLK) {CLK};

                    \draw[->] (ucsyndrome.End) -- +(3, 0) node [anchor=west] {end\_syndrome};
                    \draw[->] (ucsyndrome.Calc) -- +(1, 0) node [anchor=west] {} -- +(1, -1.25) -- +(-6, -1.25) |- (utsyndrome.calc);
                    \draw[->] (ucsyndrome.Ld) -- +(1.25, 0) node [anchor=west] {} -- +(1.25, -2) -- +(-5.75, -2) |- (utsyndrome.ld);
                    \draw[->] (ucsyndrome.Clear) -- +(0.75, 0) node [anchor=west] {} -- +(0.75, -0.5) -- +(-6.25, -0.5) |- (utsyndrome.clear);

                    \draw[<-, double] (utsyndrome.Data) -- +(-3, 0) node [anchor=east] {Data\_in (FifoOut)};
                    \draw[<-] (utsyndrome.calc) -- +(-1, 0) node [anchor=east] {};
                    \draw[<-] (utsyndrome.CLK) -- +(-2, 0) node [anchor=east] {} |- (CLK);

                    \draw[->, double] (utsyndrome.Syndrome) -- +(3, 0) node [anchor=west] {syndrome};

                    \draw[dashed] (0,0) -- (10, 0) -- (10, 7) -- (0, 7) -- (0,0);
                \end{tikzpicture}

                \caption{Connexions entre l'UC et l'UT du calcul du syndrome}
                \label{fig:me_syndrome}
            \end{figure}


        \section{Look up table}
            \subsection{Unité de traitement}
            \begin{figure}[H]
                \centering
                \begin{tikzpicture}[>=stealth, node distance=3cm, scale=0.7, every node/.style={scale=0.7}, circuit logic US]
                    \tikzstyle{connectionpoint} = [circle, draw, fill=black, scale=0.5];
                    %\draw[help lines] (0, 0) grid (20,20);

                    \node[shape=counter] at (5,11) (p1counter) {P1};
                    \node[shape=counter] at (5,5) (p2counter) {P2 + 1};

                    \node[shape=lutcomponent] at (9, 8) (lut) {LUT};

                    \node[shape=dff] at (12, 14) (syn) {};

                    \node[xor gate, scale=1.5] at (12, 8) (xor) {};

                    \node[circle, draw] at (14, 9) (cmp1) {=};
                    \node[circle, draw] at (14, 7) (cmp2) {=};

                    \node[circle, draw] at (9, 11) (p1max) {=};
                    \node[circle, draw] at (9, 5) (p2max) {=};

                    \draw[<-, double] (syn.D) -- +(-9.5, 0) node [anchor=east] {Syndrome $[9:0]$};
                    \draw[<-] (syn.S) -- +(0, -0.25) -- +(-0.75, -0.25) |- (2, 13.5) node [anchor=east] {Ld\_syndrome};

                    \draw[<-] (p1counter.Inc) -- +(-2, 0) node [anchor=east] {Inc\_P1};
                    \draw[<-] (p1counter.Raz) -- +(-2, 0) node [anchor=east] {Raz};

                    \draw[<-] (p2counter.Inc) -- +(-2, 0) node [anchor=east] {Inc\_P2};
                    \draw[<-] (p2counter.Ld) -- +(-2, 0) node [anchor=east] {Ld\_P2};
                    \draw[<-] (p2counter.Raz) -- +(-2, 0) node [anchor=east] {Raz};

                    \draw[->, double] (p1counter.Q) -- +(1, 0) -- +(1, -4) -- +(-3, -4) |- (p2counter.Data);
                    \draw[->, double] (p1counter.Q) -- +(1, 0) -- +(1, -4) |- (lut.Pone);
                    \draw[->, double] (p1counter.Q) -- +(1, 0) |- (p1max.west);
                    \draw[->, double] (p1counter.Q) -| +(1, -2) -- +(10.25, -2) node[anchor=west] {P1};

                    \draw[->, double] (p2counter.Q) -- +(1, 0) |- (lut.Ptwo);
                    \draw[->, double] (p2counter.Q) -- +(1, 0) |- (p2max.west);
                    \draw[->, double] (p2counter.Q) -- +(10.25, 0) node[anchor=west] {P2};

                    \draw[->, double] (lut.Sone) -- +(1, 0) |- (xor.input 1);
                    \draw[->, double] (lut.Sone) -- +(1, 0) |- (cmp1.west);
                    \draw[->, double] (lut.Stwo) -- +(1, 0) |- (xor.input 2);

                    \draw[->, double] (xor.output) -- +(0.25, 0) |- (cmp2.west);

                    \draw[->, double] (syn.Q) -| (cmp1.north);
                    \draw[->, double] (syn.Q) -| (13.33, 7.5) -| (cmp2.north);

                    \draw[->] (cmp1.east) -- +(2, 0) node [anchor=west] {ERR1};
                    \draw[->] (cmp2.east) -- +(2, 0) node [anchor=west] {ERR2};
                    \draw[->] (p1max.east) -- +(7, 0) node [anchor=west] {P1\_max};
                    \draw[->] (p2max.east) -- +(7, 0) node [anchor=west] {P2\_max};
                    \draw[<-, double] (p1max.north) -- +(0, 1) node [anchor=south] {30};
                    \draw[<-, double] (p2max.south) -- +(0, -1) node [anchor=north] {31};

                    \node at (6, 12)[anchor=south west] {5};
                    \node at (6, 6)[anchor=south west] {5};
                    \node at (10, 8.5)[anchor=south west] {10};
                    \node at (10, 7.5)[anchor=south west] {10};

                    \draw[-,dashed] (2.5,2) -- (15.5, 2) -- (15.5, 16) -- (2.5, 16) -- (2.5, 2);

                    \node[connectionpoint] at (7, 11) {};
                    \node[connectionpoint] at (7, 10) {};
                    \node[connectionpoint] at (7, 8.5) {};
                    \node[connectionpoint] at (7, 6) {};
                    \node[connectionpoint] at (11, 8.5) {};
                    \node[connectionpoint] at (13.3, 14.4) {};

                \end{tikzpicture}
                \caption{Unité de traitement de la Look up table.}
                \label{fig:me_syndrome}
            \end{figure}

            \subsection{Unité de controle}

            \begin{figure}[H]
                \begin{tikzpicture}[>=stealth',shorten >=1pt,auto,node distance=4cm]
                    \tikzstyle{every initial by arrow} = [initial text=reset, text=red, -, draw=red, decorate, decoration=zigzag]
                    \node[state, accepting, initial](E1){Idle};
                    \node[state](E2)[below of=E1]{Look up};

                    \path[->] (E1) edge [loop right] node {!start\_lut} (E1);
                    \path[->] (E1) edge node {start\_lut / raz, ld\_syndrome} (E2);

                    \path[->] (E2) edge [loop right] node  {p2\_max / inc\_p1, ld\_p2} (E2);
                    \path[->] (E2) edge [loop below] node {!p1\_max . !p2\_max / inc\_p2} (E2);

                    \draw[->] (node cs:name=E2,angle=205) parabola +(-1, 1.5) node[anchor=east] {err1 / end\_lut, ld\_err, err = "01"} parabola[bend at end] (node cs:name=E1,angle=250);
                    \draw[->] (node cs:name=E2,angle=180) parabola +(-1, 2) node[anchor=east] {err2 / end\_lut, ld\_err, err = "10"} parabola[bend at end] (node cs:name=E1,angle=225);
                    \draw[->] (node cs:name=E2,angle=155) parabola +(-1, 2.5) node[anchor=east] {p1\_max / end\_lut, ld\_err, err = "11"} parabola[bend at end] (node cs:name=E1,angle=200);
                    %\draw[-] (node cs:name=E2, angle=100) edge [bend left] node[pos=0.7, left] {err1 / end\_lut, ld\_err, err = "01"} (node cs:name=E1, angle=250);
                    %\draw[-] (node cs:name=E2, angle=125) edge [bend left] node[pos=0.5, left] {err2 / end\_lut, ld\_err, err = "10"} (node cs:name=E1, angle=225);
                    %\draw[-] (node cs:name=E2, angle=150) edge [bend left] node[pos=0.3, left] {p1\_max / end\_lut, ld\_err, err = "11"} (node cs:name=E1, angle=200);

                \end{tikzpicture}
                \caption{Machine à état utilisée pour le calcul du syndrome}
                \label{fig:me_syndrome}
            \end{figure}

            \subsection{Définition des connexions entre l'UC et l'UT}
            \begin{figure}[H]
                \centering
                \begin{tikzpicture}[>=stealth, scale=0.7, every node/.style={scale=0.7}, circuit logic US]
                    %\draw[help lines] (0, 0) grid (20,20);
                    \node[shape=uclut] at (5, 11) (uclut) {UC\_Lut};
                    \node[shape=utlut] at (5, 5) (utlut) {UT\_Lut};

                    \draw[->] (uclut.Raz) -- (utlut.Raz);
                    \draw[->] (uclut.LdPtwo) -- (utlut.LdPtwo);
                    \draw[->] (uclut.LdSyndrome) -- (utlut.LdSyndrome);
                    \draw[->] (uclut.IncPone) -- (utlut.IncPone);
                    \draw[->] (uclut.IncPtwo) -- (utlut.IncPtwo);

                    \draw[->] (utlut.PoneMax) -- (uclut.PoneMax);
                    \draw[->] (utlut.PtwoMax) -- (uclut.PtwoMax);
                    \draw[->] (utlut.ErrOne) -- (uclut.ErrOne);
                    \draw[->] (utlut.ErrTwo) -- (uclut.ErrTwo);

                    \draw[<-] (uclut.Start) -- +(-2, 0) node[anchor=east] {Start\_lut};
                    \draw[<-] (uclut.Reset) -- +(-2, 0) node[anchor=east] {Reset};

                    \draw[->] (uclut.End) -- +(2, 0) node[anchor=west] {End\_lut};
                    \draw[->, double] (uclut.Err) -- +(2, 0) node[anchor=west] {Err};
                    \draw[->] (uclut.LdErr) -- +(2, 0) node[anchor=west] {Ld\_Err};

                    \draw[->] (utlut.Pone) -- +(2, 0) node[anchor=west] {P1};
                    \draw[->] (utlut.Ptwo) -- +(2, 0) node[anchor=west] {P2};

                    \draw[<-, double] (utlut.Syndrome) -- +(-2,0) node[anchor=east] {Syndrome};

                    \node at (7.5, 11.6)[anchor=south west] {2};
                    \node at (2.5, 5)[anchor=south east] {10};

                    \draw[-,dashed] (1.5, 1.5) -- (8.5, 1.5) -- (8.5, 14.5) -- (1.5, 14.5) -- (1.5, 1.5);

                \end{tikzpicture}
                \caption{Connexions entre l'UC et l'UT de la look up table}
                \label{fig:me_syndrome}
            \end{figure}


        \section{Correcteur}
            \subsection{Unité de traitement}
            Le correcteur est basé sur un registre à décalage laissant passer
            un à un les 21 bits d'information du message. Ceux-ci seront
            corrigés, c'est à dire inversés, si leur position correspond avec
            les valeurs P1 ou P2, à condition que des erreurs soient
            effectivement présentes. La position du bit en cours de décalage
            est donnée par un registre de comptage, qui permet également à
            l'unité de contrôle de réaliser précisément 21 décalages. Le format
            de sortie désiré est par la suite créé et le nouveau message ainsi
            constitué sera maintenu jusqu'à ce qu'un autre cycle de la Look Up
            Table soit complété. Le schéma logique présentant ce foncitonnent
            est donné dans la figure~\ref{fig:ut_corr}

            \begin{figure}[H]
                \centering
                \begin{tikzpicture}[>=stealth, scale=0.6, every node/.style={scale=0.6}, circuit logic US]
                    \tikzstyle{connectionpoint} = [circle, draw, fill=black, scale=0.5];
                    %\draw[help lines] (0, 0) grid (20,20);

                    \node[shape=srr] at (3, 3) (srr) {};
                    \node[not gate] at (5, 2) (not) {};

                    \node[trapezium, draw, rotate=-90, scale=3] at (10, 2.5) (muxcorr) {};
                    \draw (muxcorr.south west) -- +(0,0) node[anchor=west] {0};
                    \draw (muxcorr.south east) -- +(0,0) node[anchor=west] {1};

                    \node[shape=counter] at (3, 10) (counter) {};

                    \node[circle, draw] at (6, 12) (cmp2) {=};
                    \node[circle, draw] at (6, 8) (cmp1) {=};

                    \node[trapezium, draw, rotate=-90, scale=3] at (9, 12) (muxcmp2) {};
                    \draw (muxcmp2.south west) -- +(0,0) node[anchor=west] {1};
                    \draw (muxcmp2.south east) -- +(0,0) node[anchor=west] {0};

                    \node[trapezium, draw, rotate=-90, scale=3] at (9, 8) (muxcmp1) {};
                    \draw (muxcmp1.south west) -- +(0,0) node[anchor=west] {0};
                    \draw (muxcmp1.south east) -- +(0,0) node[anchor=west] {1};

                    \node[or gate, rotate=-90] at (10, 6) (orcorr) {};
                    \node[or gate, rotate=-90] at (9, 14) (orsel) {};

                    \node[shape=pile] at (13, 2) (pile) {};

                    \draw[<-] (counter.Raz) -- +(-2, 0) node[anchor=east] {Raz};
                    \draw[<-, double] (srr.Din) -- +(-2, 0) node[anchor=east] {Din};
                    \draw[<-] (srr.Ld) -- +(-2, 0) node[anchor=east] {Ld\_buf};
                    \draw[<-] (srr.Shift) -- +(-2, 0) node[anchor=east] {Dec\_Buf};
                    \draw[<-] (pile.Raz) -| +(-0.25, -1) -- +(-12, -1) node[anchor=east] {Raz};
                    \draw[<-] (pile.Ld) -| +(-0.5, -1.25) -- +(-12, -1.25) node[anchor=east] {Ld\_Corr};
                    \draw[<-, double] (16, 5) |- (0, 15) node[anchor=east] (err) {Err};
                    \draw[<-] (orsel.input 1) -- +(0, 0.75);
                    \draw[<-] (orsel.input 2) -- +(0, 0.75);
                    \draw[<-] (muxcmp1.west) |- +(-2, 0.5) -- +(-2, 6.3);
                    \draw[<-, double] (cmp1.west) -| +(-0.65, -1) -- +(-5.65, -1)  node[anchor=east] {P1};
                    \draw[<-, double] (cmp2.west) -| +(-0.65, 1) -- +(-5.65, 1)  node[anchor=east] {P2};

                    \draw[->] (srr.Dout) -- +(5, 0) |- (muxcorr.south west);
                    \draw[->] (srr.Dout) -- +(0.5, 0) |- (not.input);
                    \draw[->] (not.output) -- +(3.55, 0) |- (muxcorr.south east);
                    \draw[->] (muxcorr.north) -- +(1, 0) |- (pile.In);

                    \draw[->] (muxcmp1.north) -| (orcorr.input 2);
                    \draw[->] (muxcmp2.north) -| (orcorr.input 1);
                    \draw[->] (orcorr.output) -- (muxcorr.west);

                    \draw[<-] (muxcmp1.south west) -- +(-1, 0) node[anchor=east] {0};
                    \draw[->] (cmp1.east) -- +(2, 0) |- (muxcmp1.south east);

                    \draw[<-] (muxcmp2.south east) -- +(-1, 0) node[anchor=east] {0};
                    \draw[->] (cmp2.east) -- +(2, 0) |- (muxcmp2.south west);

                    \draw[->] (orsel.output) -- (muxcmp2.west);
                    \draw[->, double] (counter.Q) -| (cmp2.south);
                    \draw[->, double] (counter.Q) -| (cmp1.north);
                    \draw[->, double] (counter.Q) -- +(2, 0) node [connectionpoint] {} -- +(16, 0) node[anchor=west] {Count};

                    \draw[<-] (counter.Inc) -- +(-0.5, 0) node[anchor=east] {1};

                    \draw[->, double] (14, 5) -- +(6, 0) node[anchor=west] {D\_corr\_out};

                    \draw[->, double] (pile.Dout) -| +(4, 3);

                    \draw[->, double] (17, 6) node[anchor=south] {0} -- +(0, -1);
                    \draw[->, double] (15, 6) node[anchor=south] {0} -- +(0, -1);

                    \node at (1, 15)[anchor=south west] {2};
                    \node at (1, 13)[anchor=south west] {5};
                    \node at (1, 7)[anchor=south west] {5};
                    \node at (1, 3.75)[anchor=south west] {32};
                    \node at (4, 11)[anchor=south west] {5};
                    \node at (9, 15)[anchor=north west] {[1]};
                    \node at (9, 15)[anchor=north east] {[0]};
                    \node at (7, 15)[anchor=north west] {[1]};
                    \node at (14, 2)[anchor=south west] {32};

                    \node at (15, 5)[anchor=south west] {\rotatebox{90}{6 $[31:26]$}};
                    \node at (16, 5)[anchor=south west] {\rotatebox{90}{2 $[25:24]$}};
                    \node at (17, 5)[anchor=south west] {\rotatebox{90}{3 $[23:21]$}};
                    \node at (18, 5)[anchor=north west] {\rotatebox{-90}{21 $[20:0]$}};

                    \draw[dashed] (1, 0) -- (19, 0) -- (19, 16) -- (1, 16) -- (1,0);


                \end{tikzpicture}
                \caption{Unité de traitement du correcteur}
                \label{fig:ut_corr}
            \end{figure}

            \subsection{Unité de contrôle}
            \begin{figure}[H]
                \centering
                \begin{tikzpicture}[>=stealth',shorten >=1pt,auto,node distance=4cm]
                    \tikzstyle{every initial by arrow} = [initial text=reset, text=red, -, draw=red, decorate, decoration=zigzag]

                    \node[state, accepting, initial](E1){Idle};
                    \node[state](E2)[below of=E1]{Correction};
                    \node[right of=E1, right=-3.5cm]{Raz};

                    \path[->] (E1) edge [loop above] node {!start\_corr} (E1);
                    \path[->] (E1) edge node {start\_corr / ld\_buff} (E2);

                    \path[->] (E2) edge [loop right] node {Count != 21 / Dec\_buf, Ld\_corr} (E2);
                    \path[->] (E2) edge [bend left] node {Count = 21 / end\_corr} (E1);

                \end{tikzpicture}
                \caption{Machine à état utilisée pour la correction d'erreur}
                \label{fig:me_syndrome}
            \end{figure}

            \subsection{Définition des connexions entre l'UC et l'UT}

            \begin{figure}[H]
                \centering
                \begin{tikzpicture}[>=stealth, scale=0.6, every node/.style={scale=0.6}, circuit logic US]
                    %\draw[help lines] (0, 0) grid (20,20);

                    \node[shape=uccorr] at (5, 11) (uccorr) {UC\_Corr};
                    \node[shape=utcorr] at (5, 5) (utcorr) {UT\_Corr};

                    \draw[->](uccorr.Raz) -- (utcorr.Raz);
                    \draw[->](uccorr.DecBuf) -- (utcorr.DecBuf);
                    \draw[->](uccorr.LdBuf) -- (utcorr.LdBuf);
                    \draw[->](uccorr.LdCorr) -- (utcorr.LdCorr);

                    \draw[->](utcorr.Count) -- (uccorr.Count);

                    \draw[<-](uccorr.Start) -- +(-2, 0) node[anchor=east] {Start\_Corr};
                    \draw[<-](uccorr.Reset) -- +(-2, 0) node[anchor=east] {Reset};

                    \draw[->](uccorr.End) -- +(2, 0) node[anchor=west] {End\_corr};

                    \draw[<-,double](utcorr.Data) -- +(-2, 0) node[anchor=east] {FifoOut};
                    \draw[<-,double](utcorr.Err) -- +(-2, 0) node[anchor=east] {Err};
                    \draw[<-](utcorr.Pone) -- +(-2, 0) node[anchor=east] {P1};
                    \draw[<-](utcorr.Ptwo) -- +(-2, 0) node[anchor=east] {P2};
                    \draw[<-](utcorr.Reset) -- +(-2, 0) node[anchor=east] {Reset};
                    \draw[->](utcorr.DCorrOut) -- +(2, 0) node[anchor=west] {D\_Corr\_Out};

                    \draw[dashed](1.5, 1.5) -- (8.5, 1.5) -- (8.5, 14.5) -- (1.5, 14.5) -- (1.5, 1.5);

                \end{tikzpicture}
                \caption{Définition des connexions entre l'UC et l'UT du correcteur}
                \label{fig:me_syndrome}
            \end{figure}

        \section{Interface Avalon}
            \subsection{Vue extérieure}
            \begin{figure}[H]
                \centering
                \begin{tikzpicture}[>=stealth, scale=1, every node/.style={scale=1}, circuit logic US]

                    \node[shape=avalon] at (5, 5) (avalon) {Avalon};

                    \draw[<-] (avalon.Addr) -- +(-2, 0) node[anchor=east] {Addr};
                    \draw[<-] (avalon.R) -- +(-2, 0) node[anchor=east] {R};
                    \draw[<-] (avalon.W) -- +(-2, 0) node[anchor=east] {W};
                    \draw[<-] (avalon.Din) -- +(-2, 0) node[anchor=east] {D\_in};
                    \draw[->] (avalon.Dout) -- +(-2, 0) node[anchor=east] {D\_out};
                    \draw[->] (avalon.Irqn) -- +(-2, 0) node[anchor=east] {Irq\_n};
                    \draw[<-] (avalon.Reset) -- +(-2, 0) node[anchor=east] {Reset};

                    \draw[<-] (avalon.AskIrq) -- +(2, 0) node[anchor=west] {Ask\_irq};
                    \draw[<-] (avalon.CorrOut) -- +(2, 0) node[anchor=west] {CorrOut};
                    \draw[<-] (avalon.CorrOutLd) -- +(2, 0) node[anchor=west] {Corr\_out\_ld};
                    \draw[->] (avalon.FifoOut) -- +(2, 0) node[anchor=west] {FifoOut};
                    \draw[->] (avalon.Words) -- +(2, 0) node[anchor=west] {Words};
                    \draw[->] (avalon.Decode) -- +(2, 0) node[anchor=west] {Decode};

                \end{tikzpicture}
                \caption{Vue extérieur du block avalon}
                \label{fig:me_syndrome}
            \end{figure}

            \subsection{schéma fonctionnel}
            \begin{figure}[H]
                \centering
                \begin{tikzpicture}[>=stealth, scale=0.5, every node/.style={scale=0.5}, circuit logic US]
                    \tikzstyle{connectionpoint} = [circle, draw, fill=black, scale=0.5];
                    %\draw[help lines] (0, 0) grid (30,40);

                    % Major busses

                    \foreach [count=\ni] \txt in {Addr, R, W, Din, Dout, Irq\_n, Reset}{
                        \draw[-] (-0.5 + 0.5*\ni, 0) -- (-0.5 + 0.5*\ni, 40) node[anchor=base] {\rotatebox{90}{\txt}};
                    }

                    \foreach [count=\ni] \txt in {Decod, Words, FifoOut, CorrOutLd, CorrOut, Ask\_irq}{
                        \draw[-] (30.5 - 0.5*\ni, 0) -- (30.5 - 0.5*\ni, 40) node[anchor=base] {\rotatebox{90}{\txt}};
                    }

                    \foreach [count=\ni] \txt in {Status, Control, Data}{
                        \draw[-] (9.5 + 0.5*\ni, 0) -- (9.5 + 0.5*\ni, 40) node[anchor=base] {\rotatebox{90}{\txt}};
                    }

                    \node[shape=decaddr] at (7, 37) (decaddr) {Dec Addr};
                    \draw[-] (decaddr.Addr) -- +(-5, 0) node[connectionpoint] {};
                    \draw[-] (decaddr.Status) -- +(1, 0) node[connectionpoint] {};
                    \draw[-] (decaddr.Control) -- +(1.5, 0) node[connectionpoint] {};
                    \draw[-] (decaddr.Data) -- +(2, 0) node[connectionpoint] {};

                    \node[shape=fifo] at (16, 30) (fifo) {Fifo};
                    \draw[-] (fifo.Dout) -- +(11, 0) node[connectionpoint] {};
                    \draw (fifo.Init) -- +(-11, 0) node[connectionpoint] {};

                    \node[trapezium, draw, rotate=-180, scale=3] at (13, 33) (muxfifo) {};
                    \draw (muxfifo.south west) -- +(0,0) node[anchor=north] {1};
                    \draw (muxfifo.south east) -- +(0,0) node[anchor=north] {0};
                    \draw (muxfifo.north) |- (fifo.Din);
                    \draw (muxfifo.south east) |- +(-11.1, 1) node[connectionpoint] {};
                    \draw (muxfifo.south west) |- +(14.6, 1) node[connectionpoint] {};
                    \draw (muxfifo.west) -- +(14.9, 0) node[connectionpoint] {};

                    \node[or gate, rotate=90] at (12, 28) (orfifow) {};
                    \node[or gate, rotate=90] at (13, 27) (orfifor) {};
                    \draw (fifo.W) -| (orfifow.output);
                    \draw (fifo.R) -| (orfifor.output);

                    \node[and gate] at (11.5, 26) (andfifow) {};
                    \node[and gate] at (12.5, 25) (andfifor) {};

                    \draw (orfifow.input 1) |- (andfifow.output);
                    \draw (orfifow.input 2) |- +(16.45, -0.25) node[connectionpoint] {};
                    \draw (orfifor.input 1) |- (andfifor.output);
                    \draw (orfifor.input 2) |- +(15.45, -0.25) node[connectionpoint] {};
                    \draw (andfifow.input 1) -- +(-0.27, 0) node[connectionpoint] {};
                    \draw (andfifow.input 2) -- +(-10.27, 0) node[connectionpoint] {};
                    \draw (andfifor.input 1) -- +(-1.37, 0) node[connectionpoint] {};
                    \draw (andfifor.input 2) -- +(-11.8, 0) node[connectionpoint] {};

                    \node[shape=dff] at(15, 22) (irqen) {IrqEn};
                    \node[and gate] at(14, 21) (andirqen) {};
                    \draw (andirqen) -| (irqen.S);
                    \draw (andirqen.input 1) -- +(-12.75, 0) node [connectionpoint] {};
                    \draw (andirqen.input 2) -- +(-3.35, 0) node [connectionpoint] {};
                    \draw (irqen.D) -- +(-13, 0) node [connectionpoint] {};
                    \draw (irqen.D) -- +(0,0) node[anchor=south east] {$[1]$};

                    \node[shape=dff] at(15, 19) (irq) {Irq};
                    \node[and gate] at (14, 20) (andirq) {};
                    \draw (andirq.output) -| (irq.R);
                    \draw (andirq.input 1) -- +(-13.3, 0) node [connectionpoint] {};
                    \draw (andirq.input 2) -- +(-3.85, 0) node [connectionpoint] {};
                    \draw (irq.D) -- +(-1, 0) node[anchor=east] {1};
                    \draw (irq.S) |- +(12.5, -0.25) node[connectionpoint] {};

                    \node [or gate, rotate=90] at (16, 23.5) (andirqn) {};
                    \draw (andirqn.input 1) |- (irqen.Qn);
                    \draw (andirqn.input 2) |- (irq.Qn);
                    \draw (andirqn.output) |- +(-13.5, 0.25) node [connectionpoint] {};

                    \node [shape=dff] at (15, 8) (decode) {Dec};
                    \draw (decode.Q) -- +(14.5, 0) node [connectionpoint] {};
                    \draw (decode.D) -- +(-13, 0) node [connectionpoint] {};
                    \draw (decode.D) -- +(0, 0) node [anchor=south east] {$[0]$};
                    \draw (decode.R) |- +(12.5, 0.25) node [connectionpoint] {};
                    \node [and gate, rotate=90] at (15, 6.5) (anddecode) {};
                    \draw (anddecode.output) -- (decode.S);
                    \draw (anddecode.input 1) |- +(-13.95, -0.25) node [connectionpoint] {};
                    \draw (anddecode.input 2) |- +(-4.55, -0.5) node [connectionpoint] {};

                    \node [shape=counter] at (15, 3) (words) {Words};
                    \draw (words.Q) -- +(13.5, 0) node [connectionpoint] {};
                    \draw (words.Raz) -| +(-0.5, -1.75) -- +(13.45, -1.75) node [connectionpoint] {};
                    \draw (words.Inc) -- +(-1, 0) node [and gate, anchor=output] (andwords) {};
                    \draw (andwords.input 1) -- +(-1.6, 0) node [connectionpoint] {};
                    \draw (andwords.input 2) -- +(-11.6, 0) node [connectionpoint] {};

                    \node [trapezium,draw,rotate=90,scale=4] at(15, 12) (muxdout) {};
                    \draw (muxdout.south west) -- +(0,0) node [anchor=east] {1X};
                    \draw (muxdout.south) -- +(0,0) node [anchor=east] {01};
                    \draw (muxdout.south east) -- +(0,0) node [anchor=east] {00};
                    \draw (muxdout.north) -- +(-12.5, 0) node [connectionpoint] {};
                    \draw (muxdout.east) |- +(-15, 0.25) node [connectionpoint] {};
                    \draw (muxdout.south west) -- +(13.5, 0) node [connectionpoint] {};
                    \draw (muxdout.south east) -| +(1.5, 4);
                    \draw (muxdout.south) -| +(5.5, 4.5);
                    \draw[->] (irq.Qn) -- +(3, 0) |- +(1.5, -2.6);
                    \draw[->] (fifo.Empty) -- +(0.75, 0) |- +(-1, -14.25);
                    \draw[->] (fifo.Full) -- +(1, 0) |- +(-1, -15.75);
                    \draw[<-] (17, 13) -- +(1, 0) node [anchor=west] {0};
                    \draw[<-] (21, 16) -- +(9, 0) node [connectionpoint] {};
                    \draw[->] (irqen.Q) -- +(6.5, 0) |- +(5.5, -8);
                    \draw[<-] (21, 13) -- +(1, 0) node[anchor=west] {0};

                \end{tikzpicture}
                \caption{Circuit logique du wrapper avalon}
                \label{fig:me_syndrome}
            \end{figure}

    \begin{thebibliography}{99}
        %\bibitem{openlayers} Openlayers, \url{http://openlayers.org/}
    \end{thebibliography}

    \begin{appendices}
    \end{appendices}

\end{document}
