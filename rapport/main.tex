\documentclass[a4paper, 11pt, svgnames]{report}
\usepackage[margin=3cm]{geometry}
\usepackage[T1]{fontenc}
\usepackage[utf8]{inputenc}
\usepackage[french]{babel}
\usepackage{hyperref}
\usepackage[toc,acronym,nopostdot]{glossaries}
\usepackage{graphicx}
\usepackage{float}
\usepackage{minted}
\usepackage{mathtools}
\usepackage[toc, page]{appendix}
\usepackage{subcaption}
\usepackage{rotating}
\usepackage{pst-circ}

\usepackage{tikz}
\usetikzlibrary{arrows, arrows.spaced ,automata, snakes, circuits.logic.US, calc}

\tikzstyle{block} = [draw]
\tikzstyle{asyncblock} = [draw]


\pagestyle{empty}
\makeatletter

\tikzset{add font/.code={\expandafter\def\expandafter\tikz@textfont\expandafter{\tikz@textfont#1}}}
\tikzset{port labels/.style={font=\sffamily\scriptsize}}

%
% Default style for components
%

\tikzset{every srr node/.style={draw,minimum width=2cm,minimum height=2cm,inner sep=1mm,outer sep=0pt,cap=round,add font=\sffamily}}
\tikzset{every dff node/.style={draw,minimum width=1cm,minimum height=1.5cm,inner sep=1mm,outer sep=0pt,cap=round,add font=\sffamily}}
\tikzset{every syndrome node/.style={draw,minimum width=5cm,minimum height=2cm,inner sep=1mm,outer sep=0pt,cap=round,add font=\sffamily}}
\tikzset{every utsyndrome node/.style={draw,minimum width=5cm,minimum height=2cm,inner sep=1mm,outer sep=0pt,cap=round,add font=\sffamily}}
\tikzset{every ucsyndrome node/.style={draw,minimum width=5cm,minimum height=2cm,inner sep=1mm,outer sep=0pt,cap=round,add font=\sffamily}}


%
% Shift right register
%

\pgfdeclareshape{srr}{
    % The 'minimum width' and 'minimum height' keys, not the content, determine
    % the size
    \savedanchor\northeast{%
        \pgfmathsetlength\pgf@x{\pgfshapeminwidth}%
        \pgfmathsetlength\pgf@y{\pgfshapeminheight}%
        \pgf@x=0.5\pgf@x
        \pgf@y=0.5\pgf@y
    }
    % This is redundant, but makes some things easier:
    \savedanchor\southwest{%
        \pgfmathsetlength\pgf@x{\pgfshapeminwidth}%
        \pgfmathsetlength\pgf@y{\pgfshapeminheight}%
        \pgf@x=-0.5\pgf@x
        \pgf@y=-0.5\pgf@y
    }
    % Inherit from rectangle
    \inheritanchorborder[from=rectangle]

    % Define same anchor a normal rectangle has
    \anchor{center}{\pgfpointorigin}
    \anchor{north}{\northeast \pgf@x=0pt}
    \anchor{east}{\northeast \pgf@y=0pt}
    \anchor{south}{\southwest \pgf@x=0pt}
    \anchor{west}{\southwest \pgf@y=0pt}
    \anchor{north east}{\northeast}
    \anchor{north west}{\northeast \pgf@x=-\pgf@x}
    \anchor{south west}{\southwest}
    \anchor{south east}{\southwest \pgf@x=-\pgf@x}
    \anchor{text}{
        \pgfpointorigin
        \advance\pgf@x by -.5\wd\pgfnodeparttextbox%
        \advance\pgf@y by -.5\ht\pgfnodeparttextbox%
        \advance\pgf@y by +.5\dp\pgfnodeparttextbox%
    }

    % Define anchors for signal ports
    \anchor{Din}{
        \pgf@process{\northeast}%
        \pgf@x=-1\pgf@x%
        \pgf@y=0.75\pgf@y%
    }
    \anchor{Ld}{
        \pgf@process{\northeast}%
        \pgf@x=-1\pgf@x%
        \pgf@y=0.375\pgf@y%
    }
    \anchor{Shift}{
        \pgf@process{\northeast}%
        \pgf@x=-1\pgf@x%
        \pgf@y=0\pgf@y%
    }
    \anchor{CLK}{
        \pgf@process{\northeast}%
        \pgf@x=-1\pgf@x%
        \pgf@y=-.75\pgf@y%
    }
    \anchor{Dout}{
        \pgf@process{\northeast}%
        \pgf@y=0\pgf@y%
    }

    % Draw the rectangle box and the port labels
    \backgroundpath{
        % Rectangle box
        \pgfpathrectanglecorners{\southwest}{\northeast}
        % Angle (>) for clock input
        \pgf@anchor@srr@CLK
        \pgf@xa=\pgf@x \pgf@ya=\pgf@y
        \pgf@xb=\pgf@x \pgf@yb=\pgf@y
        \pgf@xc=\pgf@x \pgf@yc=\pgf@y
        \pgfmathsetlength\pgf@x{1ex} % size depends on font size
        \advance\pgf@ya by \pgf@x
        \advance\pgf@xb by \pgf@x
        \advance\pgf@yc by -\pgf@x
        \pgfpathmoveto{\pgfpoint{\pgf@xa}{\pgf@ya}}
        \pgfpathlineto{\pgfpoint{\pgf@xb}{\pgf@yb}}
        \pgfpathlineto{\pgfpoint{\pgf@xc}{\pgf@yc}}
        \pgfclosepath

        % Draw port labels
        \begingroup
        \tikzset{port labels} % Use font from this style
        \tikz@textfont
        \tikz@textfont

        \pgf@anchor@srr@Din
        \pgftext[left,base,at={\pgfpoint{\pgf@x}{\pgf@y}},x=\pgfshapeinnerxsep]{\raisebox{-0.75ex}{Din}}

        \pgf@anchor@srr@CLK
        \pgftext[left,base,at={\pgfpoint{\pgf@x}{\pgf@y}},x=\pgfshapeinnerxsep]{\raisebox{-0.75ex}{}}

        \pgf@anchor@srr@Dout
        \pgftext[right,base,at={\pgfpoint{\pgf@x}{\pgf@y}},x=-\pgfshapeinnerxsep]{\raisebox{-0.75ex}{Dout}}

        \pgf@anchor@srr@Ld
        \pgftext[left,base,at={\pgfpoint{\pgf@x}{\pgf@y}},x=\pgfshapeinnerxsep]{\raisebox{-0.75ex}{Ld}}

        \pgf@anchor@srr@Shift
        \pgftext[left,base,at={\pgfpoint{\pgf@x}{\pgf@y}},x=\pgfshapeinnerxsep]{\raisebox{-0.75ex}{Shift}}

        \endgroup
    }
}

%
% Data Flip Flip (DFF) shape
%

\pgfdeclareshape{dff}{
  % The 'minimum width' and 'minimum height' keys, not the content, determine
  % the size
    \savedanchor\northeast{%
        \pgfmathsetlength\pgf@x{\pgfshapeminwidth}%
        \pgfmathsetlength\pgf@y{\pgfshapeminheight}%
        \pgf@x=0.5\pgf@x
        \pgf@y=0.5\pgf@y
    }
  % This is redundant, but makes some things easier:
    \savedanchor\southwest{%
        \pgfmathsetlength\pgf@x{\pgfshapeminwidth}%
        \pgfmathsetlength\pgf@y{\pgfshapeminheight}%
        \pgf@x=-0.5\pgf@x
        \pgf@y=-0.5\pgf@y
    }
    % Inherit from rectangle
    \inheritanchorborder[from=rectangle]

    % Define same anchor a normal rectangle has
    \anchor{center}{\pgfpointorigin}
    \anchor{north}{\northeast \pgf@x=0pt}
    \anchor{east}{\northeast \pgf@y=0pt}
    \anchor{south}{\southwest \pgf@x=0pt}
    \anchor{west}{\southwest \pgf@y=0pt}
    \anchor{north east}{\northeast}
    \anchor{north west}{\northeast \pgf@x=-\pgf@x}
    \anchor{south west}{\southwest}
    \anchor{south east}{\southwest \pgf@x=-\pgf@x}
    \anchor{text}{
        \pgfpointorigin
        \advance\pgf@x by -.5\wd\pgfnodeparttextbox%
        \advance\pgf@y by -.5\ht\pgfnodeparttextbox%
        \advance\pgf@y by +.5\dp\pgfnodeparttextbox%
    }

    % Define anchors for signal ports
    \anchor{D}{
        \pgf@process{\northeast}%
        \pgf@x=-1\pgf@x%
        \pgf@y=.5\pgf@y%
    }
    \anchor{CLK}{
        \pgf@process{\northeast}%
        \pgf@x=-1\pgf@x%
        \pgf@y=-.5\pgf@y%
    }
    \anchor{Q}{
        \pgf@process{\northeast}%
        \pgf@y=.5\pgf@y%
    }
    \anchor{Qn}{
        \pgf@process{\northeast}%
        \pgf@y=-.5\pgf@y%
    }

    \anchor{R}{
        \pgf@process{\northeast}%
        \pgf@x=0pt%
    }
    \anchor{S}{
        \pgf@process{\northeast}%
        \pgf@x=0pt%
        \pgf@y=-\pgf@y%
    }

    % Draw the rectangle box and the port labels
    \backgroundpath{
        % Rectangle box
        \pgfpathrectanglecorners{\southwest}{\northeast}
        % Angle (>) for clock input
        \pgf@anchor@dff@CLK
        \pgf@xa=\pgf@x \pgf@ya=\pgf@y
        \pgf@xb=\pgf@x \pgf@yb=\pgf@y
        \pgf@xc=\pgf@x \pgf@yc=\pgf@y
        \pgfmathsetlength\pgf@x{1ex} % size depends on font size
        \advance\pgf@ya by \pgf@x
        \advance\pgf@xb by \pgf@x
        \advance\pgf@yc by -\pgf@x
        \pgfpathmoveto{\pgfpoint{\pgf@xa}{\pgf@ya}}
        \pgfpathlineto{\pgfpoint{\pgf@xb}{\pgf@yb}}
        \pgfpathlineto{\pgfpoint{\pgf@xc}{\pgf@yc}}
        \pgfclosepath

        % Draw port labels
        \begingroup
        \tikzset{port labels} % Use font from this style
        \tikz@textfont

        \pgf@anchor@dff@D
        \pgftext[left,base,at={\pgfpoint{\pgf@x}{\pgf@y}},x=\pgfshapeinnerxsep]{\raisebox{-0.75ex}{D}}

        \pgf@anchor@dff@Q
        \pgftext[right,base,at={\pgfpoint{\pgf@x}{\pgf@y}},x=-\pgfshapeinnerxsep]{\raisebox{-.75ex}{Q}}

        \pgf@anchor@dff@Qn
        \pgftext[right,base,at={\pgfpoint{\pgf@x}{\pgf@y}},x=-\pgfshapeinnerxsep]{\raisebox{-.75ex}{$\overline{\mbox{Q}}$}}

        \pgf@anchor@dff@R
        \pgftext[top,at={\pgfpoint{\pgf@x}{\pgf@y}},y=-\pgfshapeinnerysep]{R}

        \pgf@anchor@dff@S
        \pgftext[bottom,at={\pgfpoint{\pgf@x}{\pgf@y}},y=\pgfshapeinnerysep]{S}

        \endgroup
    }
}


%
% External syndrome block
%

\pgfdeclareshape{syndrome}{
  % The 'minimum width' and 'minimum height' keys, not the content, determine
  % the size
    \savedanchor\northeast{%
        \pgfmathsetlength\pgf@x{\pgfshapeminwidth}%
        \pgfmathsetlength\pgf@y{\pgfshapeminheight}%
        \pgf@x=0.5\pgf@x
        \pgf@y=0.5\pgf@y
    }
  % This is redundant, but makes some things easier:
    \savedanchor\southwest{%
        \pgfmathsetlength\pgf@x{\pgfshapeminwidth}%
        \pgfmathsetlength\pgf@y{\pgfshapeminheight}%
        \pgf@x=-0.5\pgf@x
        \pgf@y=-0.5\pgf@y
    }
    % Inherit from rectangle
    \inheritanchorborder[from=rectangle]

    % Define same anchor a normal rectangle has
    \anchor{center}{\pgfpointorigin}
    \anchor{north}{\northeast \pgf@x=0pt}
    \anchor{east}{\northeast \pgf@y=0pt}
    \anchor{south}{\southwest \pgf@x=0pt}
    \anchor{west}{\southwest \pgf@y=0pt}
    \anchor{north east}{\northeast}
    \anchor{north west}{\northeast \pgf@x=-\pgf@x}
    \anchor{south west}{\southwest}
    \anchor{south east}{\southwest \pgf@x=-\pgf@x}
    \anchor{text}{
        \pgfpointorigin
        \advance\pgf@x by -.5\wd\pgfnodeparttextbox%
        \advance\pgf@y by -.5\ht\pgfnodeparttextbox%
        \advance\pgf@y by +.5\dp\pgfnodeparttextbox%
    }

    % Define anchors for signal ports
    \anchor{Data}{
        \pgf@process{\northeast}%
        \pgf@x=-1\pgf@x%
        \pgf@y=.75\pgf@y%
    }
    \anchor{Start}{
        \pgf@process{\northeast}%
        \pgf@x=-1\pgf@x%
        \pgf@y=.25\pgf@y%
    }
    \anchor{Reset}{
        \pgf@process{\northeast}%
        \pgf@x=-1\pgf@x%
        \pgf@y=-.25\pgf@y%
    }
    \anchor{CLK}{
        \pgf@process{\northeast}%
        \pgf@x=-1\pgf@x%
        \pgf@y=-.75\pgf@y%
    }

    \anchor{Syndrome}{
        \pgf@process{\northeast}%
        \pgf@y=.5\pgf@y%
    }
    \anchor{End}{
        \pgf@process{\northeast}%
        \pgf@y=-.5\pgf@y%
    }

    % Draw the rectangle box and the port labels
    \backgroundpath{
        % Rectangle box
        \pgfpathrectanglecorners{\southwest}{\northeast}
        % Angle (>) for clock input
        \pgf@anchor@syndrome@CLK
        \pgf@xa=\pgf@x \pgf@ya=\pgf@y
        \pgf@xb=\pgf@x \pgf@yb=\pgf@y
        \pgf@xc=\pgf@x \pgf@yc=\pgf@y
        \pgfmathsetlength\pgf@x{0.5ex} % size depends on font size
        \advance\pgf@ya by \pgf@x
        \advance\pgf@xb by \pgf@x
        \advance\pgf@yc by -\pgf@x
        \pgfpathmoveto{\pgfpoint{\pgf@xa}{\pgf@ya}}
        \pgfpathlineto{\pgfpoint{\pgf@xb}{\pgf@yb}}
        \pgfpathlineto{\pgfpoint{\pgf@xc}{\pgf@yc}}
        \pgfclosepath

        % Draw port labels
        \begingroup
        \tikzset{port labels} % Use font from this style
        \tikz@textfont

        \pgf@anchor@syndrome@Data
        \pgftext[left,base,at={\pgfpoint{\pgf@x}{\pgf@y}},x=\pgfshapeinnerxsep]{\raisebox{-0.75ex}{Data [31:0]}}

        \pgf@anchor@syndrome@Start
        \pgftext[left,base,at={\pgfpoint{\pgf@x}{\pgf@y}},x=\pgfshapeinnerxsep]{\raisebox{-0.75ex}{Start}}

        \pgf@anchor@syndrome@Reset
        \pgftext[left,base,at={\pgfpoint{\pgf@x}{\pgf@y}},x=\pgfshapeinnerxsep]{\raisebox{-0.75ex}{Reset}}

        \pgf@anchor@syndrome@CLK
        \pgftext[left,base,at={\pgfpoint{\pgf@x}{\pgf@y}},x=\pgfshapeinnerxsep]{\raisebox{-0.75ex}{CLK}}

        \pgf@anchor@syndrome@Syndrome
        \pgftext[right,base,at={\pgfpoint{\pgf@x}{\pgf@y}},x=-\pgfshapeinnerxsep]{\raisebox{-.75ex}{Syndrome [9:0]}}

        \pgf@anchor@syndrome@End
        \pgftext[right,base,at={\pgfpoint{\pgf@x}{\pgf@y}},x=-\pgfshapeinnerxsep]{\raisebox{-.75ex}{End}}

        \endgroup
    }
}

%
% External ut_syndrome block
%

\pgfdeclareshape{utsyndrome}{
  % The 'minimum width' and 'minimum height' keys, not the content, determine
  % the size
    \savedanchor\northeast{%
        \pgfmathsetlength\pgf@x{\pgfshapeminwidth}%
        \pgfmathsetlength\pgf@y{\pgfshapeminheight}%
        \pgf@x=0.5\pgf@x
        \pgf@y=0.5\pgf@y
    }
  % This is redundant, but makes some things easier:
    \savedanchor\southwest{%
        \pgfmathsetlength\pgf@x{\pgfshapeminwidth}%
        \pgfmathsetlength\pgf@y{\pgfshapeminheight}%
        \pgf@x=-0.5\pgf@x
        \pgf@y=-0.5\pgf@y
    }
    % Inherit from rectangle
    \inheritanchorborder[from=rectangle]

    % Define same anchor a normal rectangle has
    \anchor{center}{\pgfpointorigin}
    \anchor{north}{\northeast \pgf@x=0pt}
    \anchor{east}{\northeast \pgf@y=0pt}
    \anchor{south}{\southwest \pgf@x=0pt}
    \anchor{west}{\southwest \pgf@y=0pt}
    \anchor{north east}{\northeast}
    \anchor{north west}{\northeast \pgf@x=-\pgf@x}
    \anchor{south west}{\southwest}
    \anchor{south east}{\southwest \pgf@x=-\pgf@x}
    \anchor{text}{
        \pgfpointorigin
        \advance\pgf@x by -.5\wd\pgfnodeparttextbox%
        \advance\pgf@y by -.5\ht\pgfnodeparttextbox%
        \advance\pgf@y by +.5\dp\pgfnodeparttextbox%
    }

    % Define anchors for signal ports

    \anchor{Data}{
        \pgf@process{\northeast}%
        \pgf@x=-1\pgf@x%
        \pgf@y=.8\pgf@y%
    }

    \anchor{ld}{
        \pgf@process{\northeast}%
        \pgf@x=-1\pgf@x%
        \pgf@y=.4\pgf@y%
    }

    \anchor{calc}{
        \pgf@process{\northeast}%
        \pgf@x=-1\pgf@x%
        \pgf@y=0\pgf@y%
    }

    \anchor{clear}{
        \pgf@process{\northeast}%
        \pgf@x=-1\pgf@x%
        \pgf@y=-0.4\pgf@y%
    }

    \anchor{CLK}{
        \pgf@process{\northeast}%
        \pgf@x=-1\pgf@x%
        \pgf@y=-.8\pgf@y%
    }

    \anchor{Syndrome}{
        \pgf@process{\northeast}%
        \pgf@y=0.5\pgf@y%
    }

    % Draw the rectangle box and the port labels
    \backgroundpath{
        % Rectangle box
        \pgfpathrectanglecorners{\southwest}{\northeast}
        % Angle (>) for clock input
        \pgf@anchor@utsyndrome@CLK
        \pgf@xa=\pgf@x \pgf@ya=\pgf@y
        \pgf@xb=\pgf@x \pgf@yb=\pgf@y
        \pgf@xc=\pgf@x \pgf@yc=\pgf@y
        \pgfmathsetlength\pgf@x{0.5ex} % size depends on font size
        \advance\pgf@ya by \pgf@x
        \advance\pgf@xb by \pgf@x
        \advance\pgf@yc by -\pgf@x
        \pgfpathmoveto{\pgfpoint{\pgf@xa}{\pgf@ya}}
        \pgfpathlineto{\pgfpoint{\pgf@xb}{\pgf@yb}}
        \pgfpathlineto{\pgfpoint{\pgf@xc}{\pgf@yc}}
        \pgfclosepath

        % Draw port labels
        \begingroup
        \tikzset{port labels} % Use font from this style
        \tikz@textfont

        \pgf@anchor@utsyndrome@Data
        \pgftext[left,base,at={\pgfpoint{\pgf@x}{\pgf@y}},x=\pgfshapeinnerxsep]{\raisebox{-0.75ex}{Data [31:0]}}

        \pgf@anchor@utsyndrome@ld
        \pgftext[left,base,at={\pgfpoint{\pgf@x}{\pgf@y}},x=\pgfshapeinnerxsep]{\raisebox{-0.75ex}{Ld}}

        \pgf@anchor@utsyndrome@calc
        \pgftext[left,base,at={\pgfpoint{\pgf@x}{\pgf@y}},x=\pgfshapeinnerxsep]{\raisebox{-0.75ex}{calc}}

        \pgf@anchor@utsyndrome@clear
        \pgftext[left,base,at={\pgfpoint{\pgf@x}{\pgf@y}},x=\pgfshapeinnerxsep]{\raisebox{-0.75ex}{clear}}

        \pgf@anchor@utsyndrome@CLK
        \pgftext[left,base,at={\pgfpoint{\pgf@x}{\pgf@y}},x=\pgfshapeinnerxsep]{\raisebox{-0.75ex}{CLK}}

        \pgf@anchor@utsyndrome@Syndrome
        \pgftext[right,base,at={\pgfpoint{\pgf@x}{\pgf@y}},x=-\pgfshapeinnerxsep]{\raisebox{-.75ex}{Syndrome [9:0]}}

        \endgroup
    }
}

%
% External uc_syndrome block
%

\pgfdeclareshape{ucsyndrome}{
  % The 'minimum width' and 'minimum height' keys, not the content, determine
  % the size
    \savedanchor\northeast{%
        \pgfmathsetlength\pgf@x{\pgfshapeminwidth}%
        \pgfmathsetlength\pgf@y{\pgfshapeminheight}%
        \pgf@x=0.5\pgf@x
        \pgf@y=0.5\pgf@y
    }
  % This is redundant, but makes some things easier:
    \savedanchor\southwest{%
        \pgfmathsetlength\pgf@x{\pgfshapeminwidth}%
        \pgfmathsetlength\pgf@y{\pgfshapeminheight}%
        \pgf@x=-0.5\pgf@x
        \pgf@y=-0.5\pgf@y
    }
    % Inherit from rectangle
    \inheritanchorborder[from=rectangle]

    % Define same anchor a normal rectangle has
    \anchor{center}{\pgfpointorigin}
    \anchor{north}{\northeast \pgf@x=0pt}
    \anchor{east}{\northeast \pgf@y=0pt}
    \anchor{south}{\southwest \pgf@x=0pt}
    \anchor{west}{\southwest \pgf@y=0pt}
    \anchor{north east}{\northeast}
    \anchor{north west}{\northeast \pgf@x=-\pgf@x}
    \anchor{south west}{\southwest}
    \anchor{south east}{\southwest \pgf@x=-\pgf@x}
    \anchor{text}{
        \pgfpointorigin
        \advance\pgf@x by -.5\wd\pgfnodeparttextbox%
        \advance\pgf@y by -.5\ht\pgfnodeparttextbox%
        \advance\pgf@y by +.5\dp\pgfnodeparttextbox%
    }

    % Define anchors for signal ports

    \anchor{Start}{
        \pgf@process{\northeast}%
        \pgf@x=-1\pgf@x%
        \pgf@y=0.5\pgf@y%
    }

    \anchor{Reset}{
        \pgf@process{\northeast}%
        \pgf@x=-1\pgf@x%
        \pgf@y=0\pgf@y%
    }

    \anchor{CLK}{
        \pgf@process{\northeast}%
        \pgf@x=-1\pgf@x%
        \pgf@y=-.75\pgf@y%
    }

    \anchor{End}{
        \pgf@process{\northeast}%
        \pgf@y=0.75\pgf@y%
    }

    \anchor{Calc}{
        \pgf@process{\northeast}%
        \pgf@y=-0.25\pgf@y%
    }

    \anchor{Ld}{
        \pgf@process{\northeast}%
        \pgf@y=0.25\pgf@y%
    }

    \anchor{Clear}{
        \pgf@process{\northeast}%
        \pgf@y=-0.75\pgf@y%
    }

    % Draw the rectangle box and the port labels
    \backgroundpath{
        % Rectangle box
        \pgfpathrectanglecorners{\southwest}{\northeast}
        % Angle (>) for clock input
        \pgf@anchor@ucsyndrome@CLK
        \pgf@xa=\pgf@x \pgf@ya=\pgf@y
        \pgf@xb=\pgf@x \pgf@yb=\pgf@y
        \pgf@xc=\pgf@x \pgf@yc=\pgf@y
        \pgfmathsetlength\pgf@x{0.5ex} % size depends on font size
        \advance\pgf@ya by \pgf@x
        \advance\pgf@xb by \pgf@x
        \advance\pgf@yc by -\pgf@x
        \pgfpathmoveto{\pgfpoint{\pgf@xa}{\pgf@ya}}
        \pgfpathlineto{\pgfpoint{\pgf@xb}{\pgf@yb}}
        \pgfpathlineto{\pgfpoint{\pgf@xc}{\pgf@yc}}
        \pgfclosepath

        % Draw port labels
        \begingroup
        \tikzset{port labels} % Use font from this style
        \tikz@textfont

        \pgf@anchor@ucsyndrome@Start
        \pgftext[left,base,at={\pgfpoint{\pgf@x}{\pgf@y}},x=\pgfshapeinnerxsep]{\raisebox{-0.75ex}{Start}}

        \pgf@anchor@ucsyndrome@Reset
        \pgftext[left,base,at={\pgfpoint{\pgf@x}{\pgf@y}},x=\pgfshapeinnerxsep]{\raisebox{-0.75ex}{Reset}}

        \pgf@anchor@ucsyndrome@CLK
        \pgftext[left,base,at={\pgfpoint{\pgf@x}{\pgf@y}},x=\pgfshapeinnerxsep]{\raisebox{-0.75ex}{CLK}}

        \pgf@anchor@ucsyndrome@End
        \pgftext[right,base,at={\pgfpoint{\pgf@x}{\pgf@y}},x=-\pgfshapeinnerxsep]{\raisebox{-.75ex}{End}}

        \pgf@anchor@ucsyndrome@Calc
        \pgftext[right,base,at={\pgfpoint{\pgf@x}{\pgf@y}},x=-\pgfshapeinnerxsep]{\raisebox{-.75ex}{Calc}}

        \pgf@anchor@ucsyndrome@Ld
        \pgftext[right,base,at={\pgfpoint{\pgf@x}{\pgf@y}},x=-\pgfshapeinnerxsep]{\raisebox{-.75ex}{Ld}}

        \pgf@anchor@ucsyndrome@Clear
        \pgftext[right,base,at={\pgfpoint{\pgf@x}{\pgf@y}},x=-\pgfshapeinnerxsep]{\raisebox{-.75ex}{Clear}}

        \endgroup
    }
}

\makeatother

%
% Glossaries
%

\makeglossaries

%
% Document
%

\title{Décodeur BCH}
\author{Ronan~\bsc{Le Guillou} \and Arnaud~\bsc{Levaufre} \and Bastien~\bsc{Orivel}}

\begin{document}
    \maketitle
    \tableofcontents
    \printglossaries
    \listoffigures
    \listoftables

    \chapter{Introduction}

    \chapter{Étude du circuit}
        \section{Architecture globale}


        \section{Calcul du syndrome}
            \subsection{Unité de traitement}

            \begin{figure}[H]
                \begin{tikzpicture}[>=stealth,scale=0.375, every node/.style={scale=0.5}, circuit logic US]
                    %\draw[help lines] (0, 0) grid (29, 32);

                    \node[shape=dff] at (12, 30) (dff0) {0};
                    \node[shape=dff] at (12, 27) (dff1) {1};
                    \node[shape=dff] at (12, 24) (dff2) {2};
                    \node[shape=dff] at (12, 21) (dff3) {3};
                    \node[shape=dff] at (12, 18) (dff4) {4};
                    \node[shape=dff] at (12, 15) (dff5) {5};
                    \node[shape=dff] at (12, 12) (dff6) {6};
                    \node[shape=dff] at (12, 9) (dff7) {7};
                    \node[shape=dff] at (12, 6) (dff8) {8};
                    \node[shape=dff] at (12, 3) (dff9) {9};

                    \node[circle,draw,fill=black,scale=0.3] at (7,24)  {};
                    \node[circle,draw,fill=black,scale=0.3] at (9,24)  {};
                    \node[circle,draw,fill=black,scale=0.3] at (9, 18)  {};
                    \node[circle,draw,fill=black,scale=0.3] at (9, 15)  {};
                    \node[circle,draw,fill=black,scale=0.3] at (9, 9)  {};
                    \node[circle,draw,fill=black,scale=0.3] at (9, 6)  {};
                    \node[circle,draw,fill=black,scale=0.3] at (10.5, 28.7)  {};
                    \node[circle,draw,fill=black,scale=0.3] at (10.5, 28.7)  {};
                    \node[circle,draw,fill=black,scale=0.3] at (10.5, 25.7)  {};
                    \node[circle,draw,fill=black,scale=0.3] at (10.5, 22.7)  {};
                    \node[circle,draw,fill=black,scale=0.3] at (10.5, 19.7)  {};
                    \node[circle,draw,fill=black,scale=0.3] at (10.5, 16.7)  {};
                    \node[circle,draw,fill=black,scale=0.3] at (10.5, 13.7)  {};
                    \node[circle,draw,fill=black,scale=0.3] at (10.5, 10.7)  {};
                    \node[circle,draw,fill=black,scale=0.3] at (10.5, 7.7)   {};
                    \node[circle,draw,fill=black,scale=0.3] at (10.5, 4.7)   {};
                    \node[circle,draw,fill=black,scale=0.3] at (10.5, 1.7)   {};
                    \node[circle,draw,fill=black,scale=0.3] at (10, 31.3)   {};
                    \node[circle,draw,fill=black,scale=0.3] at (10, 28.3)   {};
                    \node[circle,draw,fill=black,scale=0.3] at (10, 25.3)   {};
                    \node[circle,draw,fill=black,scale=0.3] at (10, 22.3)   {};
                    \node[circle,draw,fill=black,scale=0.3] at (10, 19.3)   {};
                    \node[circle,draw,fill=black,scale=0.3] at (10, 16.3)   {};
                    \node[circle,draw,fill=black,scale=0.3] at (10, 13.3)   {};
                    \node[circle,draw,fill=black,scale=0.3] at (10, 10.3)   {};
                    \node[circle,draw,fill=black,scale=0.3] at (10, 7.3)    {};
                    \node[circle,draw,fill=black,scale=0.3] at (10, 4.3)    {};

                    \node[circle,draw,fill=black,scale=0.3] at (13, 30.5)    {};
                    \node[circle,draw,fill=black,scale=0.3] at (13, 27.5)    {};
                    \node[circle,draw,fill=black,scale=0.3] at (13, 24.5)    {};
                    \node[circle,draw,fill=black,scale=0.3] at (13, 21.5)    {};
                    \node[circle,draw,fill=black,scale=0.3] at (13, 18.5)    {};
                    \node[circle,draw,fill=black,scale=0.3] at (13, 15.5)    {};
                    \node[circle,draw,fill=black,scale=0.3] at (13, 12.5)    {};
                    \node[circle,draw,fill=black,scale=0.3] at (13, 9.5)     {};
                    \node[circle,draw,fill=black,scale=0.3] at (13, 6.5)     {};
                    \node[circle,draw,fill=black,scale=0.3] at (13, 3.5)     {};

                    \node[xor gate, rotate=-90, scale=1.5] at (9.5, 22) (x3) {};
                    \node[xor gate, rotate=-90, scale=1.5] at (9.5, 16) (x5) {};
                    \node[xor gate, rotate=-90, scale=1.5] at (9.5, 13) (x6) {};
                    \node[xor gate, rotate=-90, scale=1.5] at (9.5, 7) (x8) {};
                    \node[xor gate, rotate=-90, scale=1.5] at (9.5, 4) (x9) {};
                    \node[xor gate, rotate=90, scale=1.5] at (8, 28) (x10) {};

                    \node[shape=srr] at (5, 24) (synbuff) {};

                    \node[left] at (0, 26) (data) {Data $[~31:0~]$};
                    \node[left] at (0, 25) (ld) {Ld};
                    \node[left] at (0, 24) (calc) {Calc};
                    \node[right] at (19, 31.5) (syndrome) {Syndrome $[~9:0~]$};
                    \node[left] at (0, 23) (clear) {Clear} ;

                    \draw [->, double] (data) -- (2.5, 26) |- (synbuff.Din);
                    \draw [->] (ld) -- (2, 25) |- (synbuff.Ld);

                    \draw [->] (synbuff.Dout) -|  (7, 26) -| (x10.input 1);
                    \draw [->] (synbuff.Dout) -- (9, 24) -| (x3.input 2);
                    \draw [->] (synbuff.Dout) -- (9, 24) -- (9, 18) -| (x5.input 2);
                    \draw [->] (synbuff.Dout) -- (9, 24) -- (9, 15) -| (x6.input 2);
                    \draw [->] (synbuff.Dout) -- (9, 24) -- (9, 9) -| (x8.input 2);
                    \draw [->] (synbuff.Dout) -- (9, 24) -- (9, 6) -| (x9.input 2);

                    \draw [->] (x10.output) |- (dff0.D);
                    \draw [->] (x9.output) |- (dff9.D);
                    \draw [->] (x8.output) |- (dff8.D);
                    \draw [->] (x6.output) |- (dff6.D);
                    \draw [->] (x5.output) |- (dff5.D);
                    \draw [->] (x3.output) |- (dff3.D);

                    \draw [->] (dff0.Q) -| (13, 28.5) -- (11, 28.5) |- (dff1.D);
                    \draw [->] (dff1.Q) -| (13, 25.5) -- (11, 25.5) |- (dff2.D);
                    \draw [->] (dff2.Q) -| (13, 22.5) -- (13, 22.5) -| (x3.input 1);
                    \draw [->] (dff3.Q) -| (13, 19.5) -- (11, 19.5) |- (dff4.D);
                    \draw [->] (dff4.Q) -| (13, 16.5) -- (13, 16.5) -| (x5.input 1);
                    \draw [->] (dff5.Q) -| (13, 13.5) -- (13, 13.5) -| (x6.input 1);
                    \draw [->] (dff6.Q) -| (13, 10.5) -- (11, 10.5) |- (dff7.D);
                    \draw [->] (dff7.Q) -| (13, 7.5) -- (13, 7.5) -| (x8.input 1);
                    \draw [->] (dff8.Q) -| (13, 4.5) -- (13, 4.5) -| (x9.input 1);
                    \draw [->] (dff9.Q) -| (13, 1.5) -- (11, 1.5) -| (x10.input 2);

                    \draw [->] (calc) -- (2, 24) -- (2, 21) -- (10.5, 21) -- (10.5, 28.7) -| (dff0.S);
                    \draw [->] (calc) -- (2, 24) -- (2, 21) -- (10.5, 21) -- (10.5, 25.7) -| (dff1.S);
                    \draw [->] (calc) -- (2, 24) -- (2, 21) -- (10.5, 21) -- (10.5, 22.7) -| (dff2.S);
                    \draw [->] (calc) -- (2, 24) -- (2, 21) -- (10.5, 21) -- (10.5, 19.7) -| (dff3.S);
                    \draw [->] (calc) -- (2, 24) -- (2, 21) -- (10.5, 21) -- (10.5, 16.7) -| (dff4.S);
                    \draw [->] (calc) -- (2, 24) -- (2, 21) -- (10.5, 21) -- (10.5, 13.7) -| (dff5.S);
                    \draw [->] (calc) -- (2, 24) -- (2, 21) -- (10.5, 21) -- (10.5, 10.7) -| (dff6.S);
                    \draw [->] (calc) -- (2, 24) -- (2, 21) -- (10.5, 21) -- (10.5, 7.7) -| (dff7.S);
                    \draw [->] (calc) -- (2, 24) -- (2, 21) -- (10.5, 21) -- (10.5, 4.7) -| (dff8.S);
                    \draw [->] (calc) -- (2, 24) -- (2, 21) -- (10.5, 21) -- (10.5, 1.7) -| (dff9.S);

                    \draw [->] (clear) -- (1, 23) -- (1, 20) -- (10, 20) -- (10, 31.3) -| (dff0.R);
                    \draw [->] (clear) -- (1, 23) -- (1, 20) -- (10, 20) -- (10, 28.3) -| (dff1.R);
                    \draw [->] (clear) -- (1, 23) -- (1, 20) -- (10, 20) -- (10, 25.3) -| (dff2.R);
                    \draw [->] (clear) -- (1, 23) -- (1, 20) -- (10, 20) -- (10, 22.3) -| (dff3.R);
                    \draw [->] (clear) -- (1, 23) -- (1, 20) -- (10, 20) -- (10, 19.3) -| (dff4.R);
                    \draw [->] (clear) -- (1, 23) -- (1, 20) -- (10, 20) -- (10, 16.3) -| (dff5.R);
                    \draw [->] (clear) -- (1, 23) -- (1, 20) -- (10, 20) -- (10, 13.3) -| (dff6.R);
                    \draw [->] (clear) -- (1, 23) -- (1, 20) -- (10, 20) -- (10, 10.3) -| (dff7.R);
                    \draw [->] (clear) -- (1, 23) -- (1, 20) -- (10, 20) -- (10, 7.3) -| (dff8.R);
                    \draw [->] (clear) -- (1, 23) -- (1, 20) -- (10, 20) -- (10, 4.3) -| (dff9.R);

                    \draw [->] (dff0.Q) -- (18, 30.5) node [above left](b0) {0};
                    \draw [->] (dff1.Q) -- (18, 27.5) node [above left](b1) {1};
                    \draw [->] (dff2.Q) -- (18, 24.5) node [above left](b2) {2};
                    \draw [->] (dff3.Q) -- (18, 21.5) node [above left](b3) {3};
                    \draw [->] (dff4.Q) -- (18, 18.5) node [above left](b4) {4};
                    \draw [->] (dff5.Q) -- (18, 15.5) node [above left](b5) {5};
                    \draw [->] (dff6.Q) -- (18, 12.5) node [above left](b6) {6};
                    \draw [->] (dff7.Q) -- (18, 9.5) node [above left](b7) {7};
                    \draw [->] (dff8.Q) -- (18, 6.5) node [above left](b8) {8};
                    \draw [->] (dff9.Q) -- (18, 3.5) node [above left](b9) {9};
                    \draw [->, double] (18, 3.5) |- (syndrome);

                \end{tikzpicture}
                \caption{Unité de traitement pour le calcul du syndrome}
            \end{figure}

            \subsection{Unité de contrôle}

            \begin{figure}[H]
                \begin{tikzpicture}[>=stealth',shorten >=1pt,auto,node distance=4cm]
                    \tikzstyle{every initial by arrow} = [text=red, -, draw=red, decorate, decoration=zigzag]
                    \node[state,accepting, initial, initial text=reset](E1){Repos};
                    \node[state](E2)[below of=E1]{Dec\_syn};

                    \path[->] (E1) edge [loop right] node {!start\_syndrome} (E1);
                    \path[->] (E1) edge node {start\_syndrome / ld\_syn\_buff, clear, syn\_dcpt=30} (E2);
                    \path[->] (E2) edge [loop right] node {syn\_dcpt != 0 / calc, syn\_dcpt -= 1} (E2);
                    \path[->] (E2) edge [bend left] node {syn\_dcpt == 0 / end\_syndrome} (E1);

                \end{tikzpicture}
                \caption{Machine à état utilisée pour le calcul du syndrome}
            \end{figure}

        \section{Look up table}
        \section{Correcteur}

    \begin{thebibliography}{99}
        %\bibitem{openlayers} Openlayers, \url{http://openlayers.org/}
    \end{thebibliography}

    \begin{appendices}
    \end{appendices}

\end{document}
